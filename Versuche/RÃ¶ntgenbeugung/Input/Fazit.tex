\section{\label{sec:fazit}Fazit}
In diesem Versuch haben wir uns sowohl mit Röntgenabsorption als auch mit Röntgenbeugunt beschäftigt. Zum Anfang haben wir nach langer Kalibrierung Absorptionsspektren sowohl mit also auch ohne Absorberfolie aufgenomomen, um einen Zusammenhang zwischen Winkel und Wellenlänge zu finden. Hier konnten wir schön die zuvor nur theoretisch bekannten Absoprtionskanten entdecken. Dieses neu gewonnene Wissen konnten wir daraufhin direkt zur Identifizierung eines Metalls nutzen, was sehr spannend war.\\
Im zweiten Versuchsteil haben wir dann versucht, anhand von Röntgenbeugung mehr über die Kristallstruktur von Kochsalz zu erfahren. Anhand eines aufgenommenen Diffraktogramms konnten wir die verschiedenen Peaks finden, fitten und damit die Gitterkonstante bestimmen. Ebenso haben wir uns verschiedene Modelle für die Anordnung der Atome in der Elementarzelle angeschaut und konnten nur anhand der Peakintensitäten das richtige Modell herausfinden. Da die Kristallstruktur von Kochsalz bereits in der Schule gelehrt wird, war es sehr spannend, dies endlich für uns selber herauszufinden. \\
Insgesamt hat der Versuch uns einen guten Einblick in die Arbeit eines Kristallografen geben können und hat definitiv geholfen, ein vertieftes Verständnis für die Röntgenbeugung zu entwickeln, welches zuvor nur aus Festkörperphysikvorlesungen bekannt war. Auch wenn nicht alles auf Anhieb geklappt hat, war der Versuch sehr lehrreich und informativ, hat sich gelohnt und uns insbesondere bei der Auswertung Freude bereitet.  