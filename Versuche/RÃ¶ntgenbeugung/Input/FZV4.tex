\subsection{\label{subsec:FZV4}Kristall und Kristallstruktur}
\textbf{\textit{a) Was ist ein Kristall? Wo gibt es Kristalle? Wo im menschlichen Körper
finden sich Kristalle oder kristallines Material?}}\\
$\rightarrow$Ein Kristall ist ein fester Stoff, in dem die Bausteine, wie beispielsweise Atome, Ionen oder Moleküle, 
in einer präzisen Kristallstruktur mit regelmäßiger Anordnung positioniert sind. Kristalle sind weit verbreitet 
und kommen in vielen Formen vor. Beispiele sind Mineralien in der Erdkruste, Metalle und Salze. 
Auch im menschlichen Organismus findet man Kalziumphosphat-Kristalle im Knochen, 
Hydroxylapatit-Mineralkristalle im Zahnschmelz
oder Nierenstein-Kristalle \cite{EPC}. \\  

\textbf{\textit{b) Was versteht man unter der Kristallstruktur? Wozu mag es sinnvoll sein,
die Struktur eines Materials zu kennen?}}\\
$\rightarrow$Die Kristallstruktur eines Materials beschreibt die geordnete Anordnung seiner Atome, 
Ionen oder Moleküle in einem dreidimensionalen Gitter. Sie kann durch ein mathematisches Punktgitter und
durch eine Basis (Ort der Atome/Ionen bezogen auf das Punktgitter) eindeutig beschrieben werden.
Sie ist charakteristisch für jedes Material und beeinflusst seine mechanischen, elektronischen, 
optischen und thermischen Eigenschaften. Insgesamt ist das Wissen über die Kristallstruktur eines Materials 
von entscheidender Bedeutung, um diese Eigenschaften zu verstehen, zu steuern und zu optimieren \cite{EPC}. \\

\textbf{\textit{c) Was bedeutet Symmetrie anschaulich?}}\\
$\rightarrow$Anschaulich ausgedrückt bezieht sich Symmetrie darauf, dass ein Objekt 
bestimmte gleiche oder ähnliche Eigenschaften in Bezug auf sein Aussehen oder Struktur aufweist, 
wenn es bestimmte Transformationen erfährt. 
Diese Transformationen können Drehungen, Spiegelungen oder Verschiebungen sein.\\

\textbf{\textit{d) Wie ist die kubische Symmetrie definiert und wie das kubisches Gitter? 
Wie hängen Symmetrie und Gitter zusammen?}}\\
$\rightarrow$Die kubische Symmetrie bezieht sich auf die Symmetrie einer dreidimensionalen Struktur, 
die die Symmetrieeigenschaften eines Würfels hat. In einem solchen System existieren daher vier Rotationsachsen 
mit jeweils dreifacher Symmetrie, die den vier Raumdiagonalen eines Würfels entsprechen. Eine 
$N$-fache Symmetrie resultiert in diesem Zusammenhang aus der Bestimmung des kleinsten Drehwinkels $\Phi=2\pi/N$, 
um den man die Struktur drehen kann, bis sie in sich selbst übergeht \cite{EPC}. \\
Das kubische Gitter ist ein dreidimensionales Gittersystem, bei dem die Gitterpunkte entlang der 
Kanten eines Würfels angeordnet sind. Es gibt drei Haupttypen:
\begin{enumerate}
    \item Kubisch primitives Gitter: Gitterpunkte an den Würfelecken,
    \item Kubisch raumzentriertes Gitter: Einen weiteren Gitterpunkt im Zentrum des Würfels,
    \item Kubisch flächenzentriertes Gitter: Statt im Zentrum ist in der Mitte der sechs Seitenflächen jeweils ein Gitterpunkt.
\end{enumerate}
Die Klassifikation eines Punktgitters wird durch die Menge und Art der möglichen Symmetrie-Operationen bestimmt. 
Daher ist es logisch, dass kubische Gittersysteme auch kubische Symmetrie aufweisen, da sie daher als kubisches Gitter 
klassifiziert sind \cite{EPC}. \\

\textbf{\textit{e) Wo in der Elementarzelle sitzen die Atome?}}\\
$\rightarrow$Die Elementarzelle oder Einheitszelle beinhaltet nur einen Gitterpunkt und wird von den 
Basis-Vektoren aufgespannt, die im kleinsten Volumen resultieren. Seien diese Basis-Vektoren 
mit $\mathbf{a}_{i}$ und $i\,\in\,\{1,2,3\}$ beschrieben, dann können die Positionen 
aller Atome in der Elementarzelle folglich angegeben werden
\begin{equation}
    \mathbf{r}_{\text{Atom,j}} = \sum_{i=1}^{3}\,x_{i,j}\mathbf{a}_{i}, 
\end{equation}
wobei $x_{i,j}$ zwischen 0 und 1 liegt und die Position des $j$-ten Atoms relativ zu dem eingeschlossenen Gitterpunkt 
angibt \cite{EPC}. \\ 
Die Werte von $x_{i,j}$ hängen von dem gewählten mathematischen Punktgitter, sowie dem betrachteten Kristall ab.