\subsubsection{Systematische Auslöschungen und mögliche Raumgruppen}
Die Anzahl der Formeleinheiten $Z$ pro Elementarzelle lässt sich mit der Gleichung aus der Anleitung folgendermaßen bestimmen~\cite[]{Anleitung}:
\begin{equation}\label{eq:formeleinheit}
    Z = \frac{N_A \cdot \rho_x \cdot V_{EZ}}{M_M} \approx 4
\end{equation}
Es ist $N_A$ die Avogadro-Konstante, $\rho_x = \SI{2,17}[]{\gram \per \cubic \centi\metre}$ die Dichte von Kochsalz, $V_{EZ} = a^3$ das Volumen der Elementarzelle und $M_M = \SI{58,44}[]{\gram \per \mol}$ die molare Masse des Materials~\cite[]{Anleitung}.\\
Jetzt werden die gefundenen und indizierten Peaks aus Tab.~\ref{tab:gitter} und Tab.~\ref{tab:gitter2} den Reflexklassen $hkl, 0kl, hhl $ und $00l$ in Tab.~\ref{tab:zuordnung} zugeordnet. Die angegebene Bedingung ist hierbei immer geradezahlig, d.h. $h+k = 2n$ oder $l = 2n$.

\begin{table}[h!]
    \centering
     \begin{tabular}{|c|c|c|c|} 
     \hline
     Peak-Nr. &  $h k l$ & Klasse & Bedingung\\ [0.5ex] 
     \hline\hline
     1 & (1 1 1) & $hkl$ &  $h+k, h+l, k+l$ \\
     2 & (2 0 0) & $00l$ & $l$\\
     3 & (2 2 0) & $hhl$ &  $h+l, l, h$\\
     4 & (1 1 3) & $hhl$ & $h+l, l, h$\\
     5 & (2 2 2) & $hhl$ & $h+l, l, h$\\
     6 & (0 0 4) & $00l$ & $l$\\
     7 & (3 1 3)& $hhl$ & $h+l, l, h$\\
     8 & (2 0 4) & $0kl$ & $k, l$\\
     9 & (2 2 4) & $hhl$ & $h+l, l, h$\\
     10 & (3 3 3) & $hhl$ & $h+l, l, h$\\
     11 & (4 0 4) & $hhl$ & $h+l, l, h$\\
     12 & (3 1 5) & $hkl$ & $h+k, h+l, k+l$\\
     13 & (4 2 4) & $hhl$ & $h+l, l, h$\\  [1ex]
     \hline
    
     \end{tabular}
     \caption[short]{Zuordnung der Peaks zu den Reflexklassen.}
     \label{tab:zuordnung}
\end{table}

Laut den Angaben aus der Tabelle und aus dem Skript ist die einzig mögliche Raumgruppe für Kochsalz die $Fm\overline{3}m$-Gruppe. NaCl ist also frächenzentriert und hat damit 4 Gitterpunkte in einer Elementarzelle (jeweils $\nicefrac[]{1}{8}$ an den Ecken und jeweils $\nicefrac{1}{2}$ an den Flächen), was mit dem berechneten Wert für $Z$ aus Gl.~\ref{eq:formeleinheit} übereinstimmt.\\