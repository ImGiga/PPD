\subsubsection{Systematische Auslöschungen und mögliche Raumgruppen}
Die Anzahl der Formeleinheiten $Z$ pro Elementarzelle lässt sich mit der Gleichung aus der Anleitung folgendermaßen bestimmen:
\begin{equation}\label{eq:formeleinheit}
    Z = \frac{N_A \cdot \rho_x \cdot V_{EZ}}{M_M} = 4
\end{equation}
mit $N_A$ der Avogadro-Konstante, $\rho_x = \SI{2,17}[pre-unit]{\gram \per \cubic \centi\metre}$ der Dichte von Kochsalz, $V_{EZ} = a^3$ dem Volumen der Elementarzelle und $M_M = \SI{58,44}[]{\gram \per \mol}$ der molaren Masse des Materials.\\
Jetzt werden die gefundenen und indizierten Peaks aus Tab.~\ref{tab:gitter} und Tab.~\ref{tab:gitter2} den Reflexklassen $hkl, 0kl, hhl $ und $00l$ in Tab.~\ref{tab:zuordnung} zugeordnet. Die angegebene Bedingung ist hierbei immer geradezahlig, d.h. $h+k = 2n$ oder $l = 2n$.

\begin{table}[h!]
    \centering
     \begin{tabular}{|c|c|c|c|} 
     \hline
     Peak-Nr. &  $h k l$ & Klasse & Bedingung\\ [0.5ex] 
     \hline\hline
     1 & 1 1 1 & $hkl$ &  $h+k, h+l, k+l$ \\
     2 & 2 0 0 & $00l$ & $l$\\
     3 & 2 2 0 & $hhl$ &  $h+l, l, h$\\
     4 & 1 1 3 & $hhl$ & $h+l, l, h$\\
     5 & 2 2 2 & $hhl$ & $h+l, l, h$\\
     6 & 0 0 4 & $00l$ & $l$\\
     7 & 3 1 3 & $hhl$ & $h+l, l, h$\\
     8 & 2 0 4 & $0kl$ & $k, l$\\
     9 & 2 2 4 & $hhl$ & $h+l, l, h$\\
     10 & 3 3 3 & $hhl$ & $h+l, l, h$\\
     11 & 4 0 4 & $hhl$ & $h+l, l, h$\\
     12 & 3 1 5 & $hkl$ & $h+k, h+l, k+l$\\
     13 & 4 2 4 & $hhl$ & $h+l, l, h$\\  [1ex]
     \hline
    
     \end{tabular}
     \caption[short]{Zuordnung der Peaks zu den Reflexklassen.}
     \label{tab:zuordnung}
\end{table}

Laut den Angaben aus der Tabelle und aus dem Skript ist die einzig mögliche Raumgruppe für Kochsalz die $Fm\overline{3}m$-Gruppe. NaCl ist also frächenzentriert und hat damit 4 Gitterpunkte in einer Elementarzelle, was mit dem berechneten Wert für $Z$ aus Gl.~\ref{eq:formeleinheit} übereinstimmt.\\