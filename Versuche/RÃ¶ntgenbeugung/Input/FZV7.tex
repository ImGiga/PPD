\subsection{\label{subsec:FZV7}Peak im Pulverdiffraktogramm}
\textbf{\textit{a) Was kann man aus den Peaklagen bestimmen?}}\\
$\rightarrow$

\textbf{\textit{b) Was kann man aus den Peakintensitäten bestimmen?}}\\
$\rightarrow$

\textbf{\textit{c) Welche Faktoren bestimmen das Profil? Was versteht man hierbei unter
axialer Divergenz?}}\\
$\rightarrow$

\textbf{\textit{d) Was bedeuten Atomformfaktor und Strukturfaktor anschaulich?}}\\
$\rightarrow$

\textbf{\textit{e) Wie kann man die Intensität berechnen? 
Was bedeuten in diesem Zusammenhang die Begriffe Absorptions-, Lorentz-, 
Polarisations-, Extinktions- und Flächenhäuffigkeitsfaktor?}}\\
$\rightarrow$

\textbf{\textit{f) Was beschreibt der Temperaturfaktor (Debye-Waller-Faktor) und wie
wirkt er sich auf die Intensität der Reflexe aus?}}\\
$\rightarrow$

\textbf{\textit{g) Aus welchen Komponenten setzt sich der Untergrund in einem 
Pulverdiffraktogramm zusammen? Warum ist der Untergrund bei niedrigen Winkeln
oft viel höher?}}\\
$\rightarrow$