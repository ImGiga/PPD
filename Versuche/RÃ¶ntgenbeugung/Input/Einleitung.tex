\section{\label{sec:einleitung}Einleitung}
Die Erforschung der Röntgenstrahlung hat die Physik seit ihrer Entdeckung revolutioniert, 
indem sie uns Einblicke in die Mikrostruktur von Materialien ermöglicht und eine Vielzahl 
von Anwendungen in verschiedenen Bereichen eröffnet hat. 
Röntgenstrahlen sind elektromagnetische Wellen mit einer besonderen Fähigkeit: Sie können 
Materie durchdringen und dabei Informationen über deren Struktur und Zusammensetzung liefern.\\
In diesem Praktikumsbericht konzentrieren wir uns auf die Untersuchung von Röntgenstrahlabsorption 
und -beugung. 
Zunächst werden wir die grundlegende Natur der Röntgenstrahlung erläutern und einen Überblick 
über die breite Palette von Anwendungen geben, die diese Strahlen ermöglichen. 
Von der medizinischen Diagnostik bis zur Materialanalyse sind die Einsatzmöglichkeiten vielfältig und faszinierend. \
Im weiteren Verlauf dieses Berichts werden wir uns auf den Versuchsablauf konzentrieren. 
Wir beginnen mit einer Absorptionsmessung, bei der das Röntgenspektrum einer Wolfram-Anode über ein Diffraktometer 
aufgenommen wird. 
Dabei wird die Strahlung nach der Referenzmessung durch eine Metallfolie abgeschwächt. 
Dieser Teil des Versuchs ermöglicht es uns, den Umgang mit den Messaufbauten zu erlernen und uns mit 
der Bragg-Bedingung vertraut zu machen. \\
Anschließend werden die Beugungspeaks von Kochsalz (NaCl) mit einem Pulver-Diffraktometer vermessen, 
was uns Einblicke in die Kristallstruktur und Bindungsabstände dieses Materials gibt. 
Die gewonnenen Daten werden dann mithilfe spezieller Software verarbeitet, um die Genauigkeit von 
Messungen in der Röntgenspektroskopie zu verdeutlichen und zu zeigen, wie minimale Veränderungen Effekte erzeugen können. \\
Abschließen werden die Ergebnisse ausgewertet, diskutiert und zusammengefasst, was uns einen 
guten Überblick über die experimentellen Tätigkeiten und Analysen verschafft.  