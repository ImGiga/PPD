\subsection{Bestimmen und Fitten der Peaks}
In diesem Versuchsteil wurde ein Pulverdiffaktogramm von Kochsalz (NaCl) aufgenommen. Die so aufgenommenen Daten sollen nun insbesondere anhand der Peaks analysiert werden. Zuerst werden die Peaks "händisch" mit dem Auge identifiziert. Zudem werden die Peaks mithilfe der Software Jana2020 \footnote[number]{Institute of Physics,Department of StructureAnalysis, Cukrovarnicka 10, 16253 Praha 6, Czech Republic} mit einer Pseudo-Voigt-Funktion gefittet. Als Anfangsfitwert für den Gitterparameter wird hierbei der Literaturwert $a = \SI[options]{5,64}[pre-unit]{\angstrom}$ verwendet. Peaknummer, händisch bestimme Peaklage $2\theta_{[Auge]}$, gefittet Peaklage $2\theta_{[Fit]}$ sowie die Halbwertsbreite sind in Tabelle \ref{tab:peaks} dargestellt.


\begin{table}[h]\label{tab:peaks}
    \centering
     \begin{tabular}{|c|c|c|c|} 
     \hline
     Peak-Nr. & $2\theta_{[Auge]}$ & $2\theta_{[Fit]}$ & FWHM in ° \\ [0.5ex] 
     \hline\hline
     1 & \num[options]{27,46(0,05)} & \num[options]{27,518(0,005)} & \num[options]{0,308(0,005)} \\ 
     2 & \num[options]{31,85(0,03)} & \num[options]{31,881(0,005)} & \num[options]{0,311(0,005)} \\
     3 & \num[options]{45,78(0,05)} & \num[options]{45,709(0,005)} & \num[options]{0,324(0,005)} \\
     4 & \num[options]{54,18(0,03)} & / & / \\
     5 & \num[options]{56,85(0,03)} & \num[options]{56,808(0,005)} & \num[options]{0,338(0,005)} \\ 
     6 & \num[options]{66,66(0,03)} & \num[options]{66,634(0,005)} & \num[options]{0,355(0,005)} \\
     7 & \num[options]{73,75(0,05)} & / & / \\
     8 & \num[options]{75,43(0,03)} & \num[options]{75,774(0,005)} & \num[options]{0,375(0,005)} \\
     9 & \num[options]{84,58(0,03)} & \num[options]{84,554(0,005)} & \num[options]{0,399(0,005)} \\
     10 & \num[options]{89,52(0,05)} & / & / \\
     11 & \num[options]{101,86(0,03)} & \num[options]{101,934(0,005)} & \num[options]{0,467(0,005)} \\
     12 & \num[options]{108,51(0,05)} & / & / \\
     13 & \num[options]{110,80(0,03)} & \num[options]{110,955(0,005)} & \num[options]{0,517(0,005)} \\  [1ex] 
     \hline
     \end{tabular}
\end{table}

Hierbei wird ersichtlich, dass nicht für alle Peaks ein sinnvoller Fit möglich war. Ebenso war die integrierte Intensität mit der Software nicht möglich und wird daher weggelassen. 

\subsection{Gitterparameter und Diskussion}

Nun soll für jeden erfolgreich gefitteten Peak der zugehörige Netzabstand berechnet werden. Hierbei gilt unter Annahme eines kubischen Kristallgitters:
\begin{equation}\label{eq:netzabstand}
    \frac{1}{d^2_{hkl}} = \frac{h^2 + k^2 + l^2}{a^2}
\end{equation}
mit $h, k, l$ den Millerschen Indizes und $a$ dem Gitterparameter. Mithilfe der Braggschen Beugungsbedingung
\begin{equation}
    2d\sin(\theta) = n\lambda
\end{equation}

mit $n = 1$ der Beugungsordnung und $\lambda = \SI[options]{1,5406}{\angstrom}$ der Wellenlänge der Röntgenstrahlung, kann der Gitterparameter berechnet werden:

\begin{equation}
    \sin^2(\theta) = \frac{\lambda^2}{4a^2} \cdot (h^2 + k^2 + l^2)
\end{equation}

Die Ergebnisse für den Netzabstand $d_{hkl}$, deren quadrierten Sinus des Beugungswinkels, die Konstante $\nicefrac[fontcmd]{\lambda^2}{4a^2}$ und den Gitterparameter $a$ sind in Tabelle \ref{tab:gitter} dargestellt. Ebenso werden die Millerschen Indizes für den jeweiligen Peak angegeben.

\begin{table}[h]\label{tab:gitter}
    \centering
     \begin{tabular}{|c|c|c|c|c|c|c|c|} 
     \hline
     Peak-Nr. & $2\theta_{[Fit]}$ & $d/\si[options]{\angstrom}$ & $\sin^2(\theta)$ & $\frac[fontcmd]{\lambda^2}{4a^2}$ & $(h^2+k^2+l^2)$ & $a/\si[options]{\angstrom}$ &  $h k l $ \\ [0.5ex] 
     \hline\hline
     1 & \num[options]{27,518(0,005)} & 3,2387 & 0,056568 & 0,018856 & 3 & 5,6096 & 1 1 1 \\
     2 & \num[options]{31,881(0,005)} & 2,8048 & 0,075424 & 0,018856 & 4 & 5,6096 & 2 0 0 \\
     3 & \num[options]{45,709(0,005)} & 1.9833 & 0.150849 & 0,018856 & 8 & 5,6096 & 2 2 0 \\
     5 & \num[options]{56,808(0,005)} & 1.6194 & 0.226273 & 0,018856 & 12 & 5,6096 & 2 2 2 \\ 
     6 & \num[options]{66,634(0,005)} & 1.4024 & 0.301698 & 0,018856 & 16 &  5,6096 & 0 0 4\\
     8 & \num[options]{75,774(0,005)} & 1.2543 & 0.377122 & 0,018856 & 20 &  5,6096 & 2 0 4\\
     9 & \num[options]{84,554(0,005)} & 1.1451  &  0.452546 & 0,018856 & 24 &  5,6096 & 2 2 4\\
     11 & \num[options]{101,934(0,005)} & 0.9916 & 0.603395 & 0,018856 & 32 &  5,6096 & 4 0 4\\
     13 & \num[options]{110,955(0,005)} & 0.9349 & 0.678820 & 0,018856 & 36 &  5,6096 & 4 2 4\\ [1ex] 
     \hline
     \end{tabular}
\end{table}

Wie man sieht, ist der Gitterparameter $a$ für alle Peaks gleich. Dies liegt natürlich auch an der Natur des Fits, welcher den Gitterparamter berechnet, für den die Abweichungen der gemessenen Werte von den theoretisch bewerteten Werten minimal sind. Daher ist die Mittelwertbildung hier überflüssig, eher kann man direkt den berechneten Wert für $a$ nehmen, den Jana2020 berechnet. Hiermit ergibt sich:

\begin{equation}
    a = \SI[options]{5,6096(0,0005)}{\angstrom}
\end{equation}

Hiermit können nun auch die Netzabstände mithilfe Gl.~\ref{eq:netzabstand} für die Peaks berechnet werden, bei denen kein Fit möglich war. Die Ergebnisse hierfür sind in Tabelle \ref{tab:nonfitval} dargestellt.

\begin{table}[h]\label{tab:nonfitval}
    \centering
     \begin{tabular}{|c|c|c|c|c|c|} 
     \hline
     Peak-Nr. & $2\theta_{[Auge]}$ & $d/\si[options]{\angstrom}$ & $\sin^2(\theta)$ & $(h^2+k^2+l^2)$&  $h k l $ \\ [0.5ex] 
     \hline\hline
     4 & \num[options]{54,18(0,03)} & 1.6914 & 0.207380 & 11 & 1 1 3 \\
     7 & \num[options]{73,75(0,05)} & 1.2869 & 0.360085 & 19 & 3 1 3\\
     10 & \num[options]{89,52(0,05)} &  & 1.0796 & 0.495811 & 27 & 3 3 3 \\
     12 & \num[options]{108,51(0,05)} & 0.9482 & 0.658735 & 35 & 3 1 5 \\
     \hline
     \end{tabular}
\end{table}