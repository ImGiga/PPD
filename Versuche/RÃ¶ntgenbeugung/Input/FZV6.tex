\subsection{\label{subsec:FZV6}Zentriertes Gitter}
\textbf{\textit{a) Was versteht man unter einem zentrierten Gitter? 
Diskutieren Sie kurz die verschiedenen Möglichkeiten der Gitterzentrierung 
und geben Sie auch an, wie viele Gitterpunkte dann jeweils pro Elementarzelle 
vorliegen. Geben Sie die Koordinaten dieser äquivalenten Gitterpunkte 
pro Elementarzelle an.}}\\
$\rightarrow$Ein zentriertes Gitter ist ein Kristallgitter, bei dem zusätzlich zu den Gitterpunkten an den Ecken 
der Elementarzelle (primitives Gitter) noch weitere Gitterpunkte im Inneren der Zelle vorhanden sind. 
Man unterscheidet zwischen basis-, flächen- oder raumzentrierten Gittern und der rhomboedrischen Zelle. 
Bei der Basis- und Raumzentrierung wird der Elementarzelle jeweils ein Gitterpunkt hinzugefügt. 
Dabei wird ein halber Punkt auf gegenüberliegenden Seitenflächen in der Basiszentrierung eingefügt, während 
der zusätzliche Punkt im Zentrum der Zelle bei Raumzentrierung liegt. 
In der rhomboedrischen Elementarzelle steigt die Gitterpunktzahl auf drei und für flächenzentrierte Gitter auf vier, 
wobei im Zentrum der sechs Außenflächen je ein halber Punkt beiträgt \cite{Schwarz, EPC}. \\ 
Die Koordinaten bezogen auf die durch $u,v,w$ definierten Gitterpunkte aus Gl.~\ref{eq:position} sind in Tab.~\ref{tab:zent} 
aufgelistet. Jede Gitterzentrierung erhält ein Kurzsymbol, wobei die Basiszentrierung in drei unterschiedlichen Arten auftreten kann \cite{Schwarz}.
\begin{table}[h!]
    \centering
    \caption{\label{tab:zent}Arten der verschiedenen Gitterzentrierungen mit zugehörigen Symbolen und Koordinaten der hinzukommenden Gitterpunkte
    bezogen auf die Basis des primitiven Gitters. Daten entnommen aus Ref.~\cite{Schwarz}.}
      \begin{tabular}{c|c|c}
      \rowcolor[rgb]{ .851,  .882,  .949} Gitterzentrieung & Symbol & Gitterpunkte $x_{j}, y_{j}, z_{j}$ \\
      \midrule
      \midrule
      primitiv & P     & $(0,0,0)$ \\
      basiszentriert $(\mathbf{b},\mathbf{c})$-Fläche & A     & $(0,0,0)$ und $(0, 1/2, 1/2)$ \\
      basiszentriert $(\mathbf{a},\mathbf{c})$-Fläche & B     & $(0,0,0)$ und $(1/2, 0, 1/2)$ \\
      basiszentriert $(\mathbf{a},\mathbf{b})$-Fläche & C     & $(0,0,0)$ und $(1/2, 1/2, 0)$ \\
      raumzentriert & I     & $(0,0,0)$ und $(1/2, 1/2, 1/2)$ \\
      flächenzentriert & F     & $(0,0,0)$ und $(0, 1/2, 1/2)$ und $(1/2, 0, 1/2)$ und $(1/2, 1/2, 0)$ \\
      rhomboedrische Zelle & R     & $(0,0,0)$ und $(2/3, 1/3, 1/3)$ und $(1/3, 2/3, 2/3)$ \\
      \end{tabular}
\end{table}\FloatBarrier \,\\

\textbf{\textit{b) Wie und warum entstehen bei zentrierten Gittern im Beugungsbild 
systematische Auslöschungen? Wie lauten diese Auslöschungsregeln?}}\\
$\rightarrow$Betrachtet man die Gitterpunkte als punktförmige Streuzentren, so wird durch die Streuung einer einfallenden ebenen Welle, 
eine Kugelwelle erzeugt. Bei Vorhandensein mehrerer Streuzentren in der Elementarzelle ist die Interferenz dieser gestreuten Wellen 
zu berücksichtigen. Durch Gitterzentrierung kann es also zu Verstärkung, Abschwächung oder sogar Auslöschung der Intensität verglichen
zum primitiven Gitter kommen, was auf konstruktive und destruktive Interferenz der an den Atomen gestreuten Kugelwellen zurückzuführen ist \cite{EPC}. \\ 
Die gemessene Intensität auf einem ausreichend weit entfernten Schirm kann mathematisch beschrieben werden, wodurch man einen 
Ausdruck bekommt, welcher die Interferenz der gestreuten Wellen berücksichtigt (vgl.~Struktur- und Atomfaktor). 
Hieraus lassen sich die Auslöschungsregeln herleiten, welche in Tab.~\ref{tab:reflex} für 
die verschiedenen Gitterzentrierungen aufgelistet sind. Die hierbei notwendigen Miller'schen Indizes ($h, k, l$) bieten eine eindeutige Beschreibung 
der im Kristall vorkommenden Gitterebenen. Die Regeln sagen also aus, welche Atomebenen zu destruktiver Interferenz führen \cite{Schwarz}.
\begin{table}[h!]
    \centering
    \caption{\label{tab:reflex}Mittels Miller-Indizes ausgedrückte Auslöschungsregeln für verschiedene Gitterzentrierungen. 
    Die Gitterebenen, welche durch Kombination von $h$, $k$ und $l$ eindeutig beschrieben sind, führen zur destruktiven Interferenz und löschen den Reflex, 
    welcher beim primitiven Gitter noch erkennbar ist, aus (vorausgesetzt man betrachtet einen Kristall mit nur einer Atomsorte).
    Daten entnommen aus Ref.~\cite{Schwarz}.}
      \begin{tabular}{c|c|c}
      \rowcolor[rgb]{ .851,  .882,  .949} Gitterzentrieung & Symbol & Auslöschungsregeln \\
      \midrule
      \midrule
      primitiv & P     & - \\
      basiszentriert $(mathbf{b},mathbf{c})$-Fläche & A     & k+l ungerade \\
      basiszentriert $(mathbf{a},mathbf{c})$-Fläche & B     & h+l ungerade \\
      basiszentriert $(mathbf{a},mathbf{b})$-Fläche & C     & h+k ungerade \\
      raumzentriert & I     & $h+k+l$ ungerade \\
      flächenzentriert & F     & $h,k,l$ nicht alle gerade oder ungerade \\
      rhomboedrische Zelle & R     & $-h+k+l\neq 3n$ oder $h-k+l\neq 3n$ mit $n\,\in\,\mathbb{N}$ \\
      \end{tabular}
\end{table}\FloatBarrier
  