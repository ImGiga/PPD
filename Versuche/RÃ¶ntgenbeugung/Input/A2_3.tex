\subsubsection{Kristallstruktur und Sturkturfaktor}
\label{sec:kristallstruktur}

In der Praktikumsanleitung gibt es zwei verschiedene Strukturmodelle A und B, die anhand der spektralen Intensitäten bewertet und auf ihre Richtigkeit überprüft werden sollen.
In Strukturmodell A liegt das Chloratom auf $(x,y,z) = (\frac{1}{2}, \frac{1}{2}, \frac{1}{2})$, in Strukturmodell B auf $(x,y,z) = (\frac{1}{4},\frac{1}{4},\frac{1}{4},)$. Das Natriumatom liegt in beiden Modellen auf $(x,y,z) = (0,0,0)$. Für beide Modelle gilt für den jeweiligen Besetzungsfaktor $n_j = 1$. Die Intensität für den Peak $hkl$ ist gegeben durch
\begin{equation}
    I_{hkl} \propto K \cdot A \cdot L \cdot P \cdot E \cdot H \cdot T \cdot \left|F_{hkl}\right|^2
\end{equation}
mit dem Lorentzfaktor $L(\theta)$, dem Polarisationsfaktor $P(\theta)$, dem Absorptionsfaktor $A(\theta)$, dem Strukturfaktor $\left|F_{hkl}\right|^2$, dem Temperaturfaktor $T(\theta)$, dem Extiktionsfaktor $E(\theta)$ und einem Skalierungsfaktor $K$. Für die Faktoren gilt:
\begin{equation}
    L (\theta)= \frac{1}{\sin(\theta) \cos(\theta)}
\end{equation}
\begin{equation}
    P (\theta) = \frac{1 + \cos^2(2\theta)}{2}
\end{equation}
\begin{equation}
    \left|F_{hkl}\right|^2 = F_{hkl} \cdot F_{hkl}^*
\end{equation}
mit der Strukturamplitude $F_{hkl}$
\begin{equation}
    F_{hkl} = \sum_{j = 1}^{N} \left[n_j f_j \cdot \exp \{2\pi i \left(hx_j+ky_j+lz_j\right)\}\right]
\end{equation}
und dem Atomformfaktor $f\left(\frac{\sin\left(\Theta\right)}{\lambda}\right)$
\begin{equation}
    f\left(\frac{\sin\left(\Theta\right)}{\lambda}\right) = \sum_{i=1}^4 \left[a_i \cdot \exp\left(-b_i \cdot \left(\frac{\sin(\Theta)}{\lambda}\right)^2\right)\right] + c
\end{equation}
Alle diese Größen werden nun für die gefitteten Peaks aus Kapitel~\ref{sec:peaks} berechnet. Die Ergebnisse sind in Tabelle~\ref{tab:ergebnisse} dargestellt.
\begin{table}[h!]
    \centering
\begin{tabular}{|c|c|c|c|c|c|c|c|c|c|c|c|c|}
    \hline
   Peak-Nr. & $2\Theta_[Fit]$& $hkl$ & $L$ & $P$ & $H$ & $f_{Na}$ & $f_{Cl}$ & $F_{hkl, A}$& $F_{hkl, B}$ & $\left|F_A^2\right|$ & $\left|F_A^2\right|$ \\ [0.5ex]
   \hline\hline
   1 & 27.518 & 1 1 1&4.33 & 0.89 & 1 & 6.78 & 9.45 & -2.67+0.00i & 6.78-9.45i & 7.13 & 135.41 \\
   2 & 31.881 &2 0 0& 3.79 & 0.86 & 3 & 6.11 & 8.73 & 14.84-0.00i & -2.63+0.00i & 220.21 & 6.90 \\
   3 & 45.709 &2 2 0& 2.79 & 0.74 & 3 & 4.34 & 7.35 & 11.69-0.00i & 11.69-0.00i & 136.69 & 136.69 \\
   5 & 56.808 &2 2 2& 2.39 & 0.65 & 1 & 3.40 & 6.60 & 10.00-0.00i & -3.20+0.00i & 99.92 & 10.23 \\
   6 & 66.634 &0 0 4& 2.18 & 0.58 & 3 & 2.84 & 6.01 & 8.85-0.00i & 8.85-0.00i & 78.30 & 78.30 \\
   8 & 75.774 &2 0 4& 2.06 & 0.53 & 6 & 2.48 & 5.53 & 8.01-0.00i & -3.04+0.00i & 64.09 & 9.27 \\
   9 & 84.554 &2 2 4& 2.01 & 0.50 & 3 & 2.24 & 5.12 & 7.36-0.00i & 7.36-0.00i & 54.20 & 54.20 \\
   11 &101.934 &4 0 4& 2.04 & 0.52 & 3 & 1.96 & 4.52 & 6.47-0.00i & 6.47-0.00i & 41.91 & 41.91 \\
   13 & 110.955 &4 2 4& 2.14 & 0.56 & 3 & 1.88 & 4.30 & 6.18-0.00i & -2.43+0.00i & 38.15 & 5.89 \\ [1ex]
   \hline
   \end{tabular}    
   \caption[short]{Ergebnisse der Berechnung der Strukturfaktoren für die gefitteten Peaks.}
   \label{tab:ergebnisse}
\end{table}

Für die Berechnung der Intensitäten werden nun $A$, $E$ und $T$ auf 1 gesetzt und die Intensitäten für beide Modelle auf die (2 0 0) Intensität normiert. Die Ergebnisse sind in Tabelle~\ref{tab:ergebnisse2} dargestellt.

\begin{table}[h!]
    \centering
    \begin{tabular}{|c|c|c|c|c|c|}
        \hline
        Peak-Nr. & $hkl$&$I_A$ & $I_B$ & $I_{norm,A}$ & $I_{norm,B}$ \\ [0.5ex]
        \hline\hline
       1 & 1 1 1 &27.6 & 523.6 & 0.01 & 7.77 \\
       2 & 2 0 0&2152.8 & 67.4 & 1.00 & 1.00 \\
       3 & 2 2 0&852.2 & 852.2 & 0.40 & 12.64 \\
       5 & 2 2 2&155.2 & 15.9 & 0.07 & 0.24 \\
       6 & 0 0 4&296.1 & 296.1 & 0.14 & 4.39 \\
       8 & 2 0 4&420.7 & 60.8 & 0.20 & 0.90 \\
       9 & 2 2 4&164.8 & 164.8 & 0.08 & 2.44 \\
       11 & 4 0 4&134.0 & 134.0 & 0.06 & 1.99 \\
       13 & 4 2 4&138.2 & 21.3 & 0.06 & 0.32 \\ [1ex]
       \hline
       \end{tabular}
        \caption[short]{Ergebnisse der Berechnung der Intensitäten für die gefitteten Peaks.}
        \label{tab:ergebnisse2}
\end{table}

Wie bereits erwähnt, war eine Integration der gemessenen Intensitäten sehr komlex, was eine quantitave Aussage über die Richtigkeit der Modelle nicht möglich macht. Rein qualitativ anhand der Höhe der Intensitäten, sichtbar in Abb.~\ref{fig:geglättet}, insbesondere verglichen mit dem zweiten Normierungspeak (2 0 0), lässt sich sagen, dass Modell A deutlich besser zu den Daten passt. Dies ist bereits am ersten Peak erkennbar: Im gemessenen Spektrum ist dieser deutlich kleiner als der Maximalpeak, auf den hier normiert wurde. Dies deckt sich mit den Daten aus Modell A, nicht jedoch mit Modell B, laut dem Peak 1 über 7 mal so hoch wie Peak 2 wäre. Auch die anderen Peaks zeigen eine ähnliche Tendenz. Dadurch lässt sich also klar sagen, dass das Natriumatom bei $(x,y,z) = (0,0,0)$ und das Chloratom bei $(x,y,z) = (\frac{1}{2}, \frac{1}{2}, \frac{1}{2})$ sitzt.