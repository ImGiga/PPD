\subsection{\label{subsec:FZV5}Frage 5}
\textbf{\textit{Diskutieren Sie die Vor- und Nachteile von Dunkelfeld-/Absorptionsspektroskopie
sowie Einzel-/Ensemblemessungen.}} \\
$\rightarrow$Die Dunkelfeldspektroskopie bietet den Vorteil, kontrastreiche Bilder ohne Hintergrund zu 
erzeugen und ermöglicht die detaillierte Analyse von Nanopartikeln und deren plasmonischen Eigenschaften. 
Nachteile liegen in der Anfälligkeit für Störsignale durch sehr kleine Verunreinigungen an Objektträgern 
und Mikroskopoberflächen sowie der Begrenzung bei der Untersuchung dickerer Präparate, 
da Signale unterschiedlicher Schichten überlappen können. 
Im Gegensatz dazu liefert die Absorptionsspektroskopie direkte quantitative Informationen über die 
Absorptionseigenschaften von Materialien und kann auf eine Vielzahl von Materialien angewendet werden. 
Allerdings eignet sie sich weniger für kleine Partikel oder wenig absorbierende Materialien, 
da das Signal im Hintergrund des Erregerlichts schwer erkennbar sein kann oder sogar untergehen kann \cite{FZV5}.\\
Einzelne Messungen ermöglichen eine präzise Analyse auf der Ebene einzelner Partikel und gewähren zudem Einblicke 
in mögliche Heterogenitäten innerhalb einer Sammlung. 
Allerdings kann der Zeitaufwand für solche Messungen erheblich sein, um Ergebnisse mit statistischer 
Signifikanz zu erzielen. 
Im Gegensatz dazu bieten Ensemblemessungen rasch statistisch aussagekräftige Daten, 
erfordern jedoch den Verzicht auf detaillierte Informationen zu Einzelpartikeln und Heterogenitäten.