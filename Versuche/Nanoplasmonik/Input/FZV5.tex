\subsection{\label{subsec:FZV5}Advantages and disadvantages}
\textbf{\textit{Discuss the advantages and disadvantages of dark field/absorption spectroscopy 
as well as single/ensemble measurements.}} \\
$\rightarrow$Dark field spectroscopy offers the advantage of generating high-contrast images without a 
background, allowing for the detailed analysis of nanoparticles and their plasmonic properties. 
However, it is susceptible to interference from very small impurities on slides and microscope surfaces, 
and it has limitations in examining thicker specimens as signals from different layers may overlap. \\
In contrast, absorption spectroscopy provides direct quantitative information about the absorption properties 
of materials and can be applied to a variety of substances. Nevertheless, it is less suitable for small 
particles or materials with low absorption, as the signal in the background of the incident light can be 
challenging to detect or may even be overshadowed \cite{FZV5}. \\
Single measurements enable a precise analysis at the level of individual particles and offer insights 
into possible heterogeneities within a collection. However, the time investment for such measurements can 
be substantial to achieve statistically significant results. Conversely, ensemble measurements quickly 
provide statistically meaningful data but require sacrificing detailed information about individual 
particles and heterogeneities. \\