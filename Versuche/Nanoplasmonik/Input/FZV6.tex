\subsection{\label{subsec:FZV6}Frage 6}
\textbf{\textit{Im Versuch wird zusätzlich zum hintergrundfreien Streuspektrum der Partikel auch
noch das Lampenspektrum der Beleuchtungseinheit, sowie ein Spektrum ohne Beleuchtung
aufgenommen. Wie müssen diese drei Spektren miteinander verrechnet
werden, um ein hintergrundkorrigiertes Dunkelfeldspektrum zu erhalten?}} \\
$\rightarrow$Es gibt zwei mögliche Ansätze zur Berechnung des hintergrundkorrigierten Dunkelfeldspektrums \cite{FZV5}.
Der subtraktive Ansatz zieht die Hintergrundsignale vom gemessenen Probenspektrum ab, wodurch 
absolute Intensitätsinformation erhalten bleibt. Der divisive Ansatz wird gewählt um 
Schwankungen im Hintergrundsignal oder im Lampenspektrum zu berücksichtigen, da eine normalisierte 
Intensität errechnet wird.
Da in unserem Versuch keine Information über die absolute Intensität notwendig ist und 
Schwankungen in der Belichtungsquelle durchaus möglich sind, wählen wir folgenden Ansatz
\begin{equation}
    \fbox{$I_{\text{K}} = \frac{I_{\text{P}}-I_{\text{D}}}{I_{\text{L}}-I_{\text{D}}}$},
\end{equation}
wobei $I_{\text{K}}$ das korrigierte Spektrum (in arb.~units) ist, $I_{\text{P}}$ dem gemessenen Probenspektrum entspricht 
und $I_{\text{L}}$ bzw.~$I_{\text{D}}$ die Intensität des Spektrums mit bzw.~ohne Beleuchtung angibt.
Es werden Lampen- und Probenspektrum daher zunächst vom konstanten Hintergrundsignal bereinigt und der 
Einfluss des variierenden Lampenspektrums aus der Messung heraus geteilt.  