\subsection{\label{subsec:FZV6}Frage 6}
\textbf{\textit{Im Versuch wird zusätzlich zum hintergrundfreien Streuspektrum der Partikel auch
noch das Lampenspektrum der Beleuchtungseinheit, sowie ein Spektrum ohne Beleuchtung
aufgenommen. Wie müssen diese drei Spektren miteinander verrechnet
werden, um ein hintergrundkorrigiertes Dunkelfeldspektrum zu erhalten?}}
$\rightarrow$