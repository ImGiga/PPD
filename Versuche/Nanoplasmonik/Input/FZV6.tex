\subsection{\label{subsec:FZV6}Background-corrected dark-field spectrum}
\textbf{\textit{In the experiment, in addition to the background-free scattering spectrum of the particles, 
the lamp spectrum of the illumination unit, as well as a spectrum without illumination, are also recorded. 
How should these three spectra be combined to obtain a background-corrected dark-field spectrum?}} \\
$\rightarrow$There are two possible approaches for calculating the background-corrected dark-field 
spectrum \cite{FZV5}. The subtractive approach involves subtracting background signals from the measured 
sample spectrum, preserving absolute intensity information. The divisive approach is chosen to account for 
fluctuations in the background signal or lamp spectrum, as it calculates a normalized intensity. \\
Since our experiment does not require information about absolute intensity, and fluctuations in the 
illumination source are conceivable, we opt for the following approach
\begin{equation}
    \fbox{$I_{\text{C}} = \frac{I_{\text{M}}-I_{\text{D}}}{I_{\text{L}}-I_{\text{D}}}$},
\end{equation}
where $I_{\text{C}}$ represents the corrected spectrum, 
$I_{\text{M}}$ corresponds to the measured sample spectrum, and $I_{\text{L}}$ or $I_{\text{D}}$ 
indicates the intensity of the spectrum with or without illumination. \\
Both the lamp and sample spectra are initially corrected for the constant background signal, and 
the influence of the varying lamp spectrum is divided out from the measurement. \\