\section{\label{sec:einleitung}Introduction}
Nanoparticles play a crucial role in various fields, from medicine to materials science, 
due to their unique properties at the nanoscale. 
In the realm of nanoscience, understanding and characterizing these particles become essential. 
One powerful technique for such analysis is Dark-Field Microscopy, enabling the investigation of 
particles below the diffraction limit by analyzing scattered light rather than transmitted light 
through the sample. \\
This experimental report delves into the exploration of nanoparticles using Dark-Field Microscopy. 
The method allows us to overcome the limitations of conventional microscopy, 
providing a detailed analysis of particles, especially those with sizes below the resolving 
power of standard imaging techniques. \\
The primary objective of this experiment is to comprehend the construction and operation 
of a Dark-Field Microscope. 
We gain hands-on experience in adjusting and calibrating the microscope, a valuable skill 
for advanced microscopy applications. 
The focus extends to the acquisition of scattering spectra from small nanorod samples 
through ensemble measurements. \\
During the experimental process, we systematically alter the polarization of incident 
light to capture scattering spectra, paving the way for a nuanced analysis. 
This approach facilitates insights into the effective refractive index of plasmons vibrating 
on the nanorods. 
Furthermore, the determination of resonance wavelength allows the calculation 
of the resonator's elongation. \\
In essence, this experiment offers a comprehensive exploration of plasmonics, 
providing a glimpse into the intricacies of experimental setups. 
Through hands-on involvement, we aim to unravel the unique optical properties 
of nanoparticles and their potential applications in cutting-edge research and technology. \\