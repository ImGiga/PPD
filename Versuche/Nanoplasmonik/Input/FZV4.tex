\subsection{\label{subsec:FZV4}Structural size below the imaging limit}
\textbf{\textit{Why is it possible to investigate the size or plasmonic properties of a 
particle using a dark field microscope, even though the structural size is well below the diffraction/imaging limit?}} \\
$\rightarrow$The diffraction limit describes the minimum size of structures that can be 
resolved in an optical system due to the wave nature of light, while the imaging limit 
takes into account the overall capability of an optical system for detailed reproduction, 
considering factors such as numerical aperture and imaging errors. \\
In dark field spectroscopy, the detected signal does not directly originate from the examined 
sample but is generated by the scattered light at the sample. Crucial is not the size of the 
object under investigation, but rather its scattering ability, from which particle properties 
can be derived. Thus, the size and plasmonic properties of a nanoparticle can be explored, 
even when their dimensions fall below the limits of diffraction/imaging. \\