\subsection{\label{subsec:FZV4}Frage 4}
\textbf{\textit{Warum ist es möglich mit Hilfe eines Dunkelfeldmikroskops die Größe bzw. die
plasmonischen Eigenschaften eines Partikels zu untersuchen, obwohl die Strukturgröße 
weit unterhalb des Beugungs/Abbildungslimits liegt?}} \\
$\rightarrow$Das Beugungslimit beschreibt die minimale Größe von Strukturen, 
die in einem optischen System aufgelöst werden können, aufgrund der Wellennatur des Lichts, 
während das Abbildungslimit die Gesamtfähigkeit eines optischen Systems zur Detailwiedergabe berücksichtigt, 
wobei Faktoren wie numerische Apertur und Abbildungsfehler eine Rolle spielen.
Das detektierte Signal in der Dunkelfeldspektroskopie entsteht nicht unmittelbar von der betrachteten Probe, 
sondern wird durch das gestreute Licht an der Probe erzeugt. 
Entscheidend ist nicht die Größe des untersuchten Objekts, sondern vielmehr das Streuvermögen,
aus dem sich Partikeleigenschaften ableiten lassen. 
Somit können die Größe und die plasmonischen Eigenschaften eines Nanopartikels erforscht werden, 
selbst wenn deren Dimensionen die Grenzen der Beugung/Abbildung unterschreiten.