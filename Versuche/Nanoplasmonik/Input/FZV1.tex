\subsection{\label{subsec:FZV1}Technical Application of Plasmonics}
\textbf{\textit{Research and briefly explain a possible technical application of plasmonics.}}\\
$\rightarrow$One potential technical application of plasmonics lies in the realm of plasmonic sensors. 
Plasmonic sensors leverage the interaction between electromagnetic waves 
and plasmons to detect minute changes in the environment, such as variations in refractive indices. 
A compelling example is found in the Covid-19 rapid test, where coronaviruses bind to receptors on 
nanostructures, altering the local refractive index in the vicinity of the nanoparticles. 
This leads to an optically observable resonance in absorption, allowing for an assessment of 
the presence of viruses \cite{FZV1p1}. \\
Furthermore, plasmonics can be harnessed for solar cell applications by enhancing the light-trapping 
properties of thin-film solar cells. This is achieved by utilizing metal nanoparticles to effectively 
couple light into the optical modes of the semiconductor. Through the targeted adjustment of surface 
plasmon resonance, absorption in the desired wavelength range is heightened. The excitation of surface 
plasmons results in intense scattering and field enhancement, potentially leading to photocurrent 
enhancements in both inorganic and organic solar cells \cite{FZV1}. \\