\subsection{\label{subsec:FZV1}Frage 1}
\textbf{\textit{Recherchieren und erläutern sie kurz eine mögliche technische Anwendung der Plasmonik.}}
$\rightarrow$Eine mögliche technische Anwendung der Plasmonik liegt im Bereich der Plasmonischen Sensoren. 
Plasmonische Sensoren nutzen die Wechselwirkung zwischen elektromagnetischen Wellen und Plasmonen, 
um kleinste Veränderungen in der Umgebung, wie beispielsweise Änderungen der Brechungsindizes, zu detektieren.
Ein Beispiel liefert der Covid-19 Schnelltest bei dem Coronaviren an Rezeptoren auf Nanostrukturen binden, 
was den lokalen Brechungsindex in der Umgebung der Nanopartikel verändert. 
Dies führt zu einer optisch wahrnehmbaren Resonanz in der Absorption, wodurch Aussage über das Vorhandensein
von Viren getroffen werden kann \cite{FZV1p1}. \\ 
Die Plasmonik kann zudem für Solarzellenanwendungen genutzt werden, da sie 
die Lichtfallen-Eigenschaften von Dünnschicht-Solarzellen verbessert, 
indem sie Metallnanopartikel nutzt, um Licht effektiv in die optischen Moden des Halbleiters zu koppeln. 
Durch die gezielte Einstellung der Oberflächenplasmonenresonanz wird die Absorption im gewünschten 
Wellenlängenbereich gesteigert. Die Anregung von Oberflächenplasmonen führt zu starker Streuung 
und Feldverstärkung, was zu Photostromverstärkungen in sowohl anorganischen als auch organischen 
Solarzellen führen kann \cite{FZV1}.