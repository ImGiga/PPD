\subsection{Fazit}
Bei der Messung mittels Weißlichtreflektometrie hat sich gezeigt, dass die Reflektionsmessung etwas bessere Ergebnisse liefert, wenn auch der Unterschied marginal ist. Die Werte beider Methoden lagen durchgehend sehr nah aneinander. Beide Messreihen konnten die Schubert-Gleichung verifizieren, dass bei konstanter Rotationsgeschwindigkeit die Schichtdicke mit der Konzentration abnimmt und bei konstanter Konzentration die Dicke mit steigener Rotationsgeschwindigkeit abnimmt. Insbesondere bei der Homogenität zeigt sich, dass bei der Probenpräparation möglicherweise fehlerhaft bzw.~nicht sortfältig genug gearbeitet wurde, die Heterogenität als Messgröße schwankt sehr stark. Bei zukünftigen Dünnfilmproben sollte deshalb noch genauer auf ein korrektes Vorgehen beim Spincoaten geachtet werden.