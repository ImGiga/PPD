\subsection{\label{subsec:FZV5}Beugung von Röntgenstrahlen}
\textbf{\textit{a) Was versteht man unter Beugung?}}\\
$\rightarrow$

\textbf{\textit{b) Erklären sie, welche Bedingungen Max von Laue zur Beschreibung der
Röntgenstrahlbeugung heranzog und stellen Sie kurz seinen Ansatz dar!}}\\
$\rightarrow$

\textbf{\textit{c) William Henry Bragg und sein Sohn William Lawrence Bragg verwendeten
ein komplett anderes aber äquivalentes Modell zur Beschreibung der
Röntgenstrahlbeugung. Erläutern Sie dieses! Wie lautet in diesem Kontext
die Braggsche Gleichung? Welchen physikalischen Zusammenhang
stellt sie dar? Erklären Sie diesen Zusammenhang anschaulich!}}\\
$\rightarrow$

\textbf{\textit{d) Hält man einen Einkristall in einen monochromatischen Röntgenstrahl,
so passiert normalerweise nichts. Welche Bedingung muss erfüllt sein, damit
ein Beugungsreflex entsteht? Welche beiden Parameter kann man nun
variieren, damit man an diesem Einkristall einen Beugungsreflex erfassen
kann?}}\\
$\rightarrow$