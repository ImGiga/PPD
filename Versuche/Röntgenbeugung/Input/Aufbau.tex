\section{\label{sec:aufbau}Aufbau und Durchführung}
\subsection{\label{subsec:vers1}Röntgenstrahlabsorption}
Im ersten Teil des Versuchs charakterisieren wir das Material einer absorbierenden Metallfolie anhand 
der Absorptionskanten, die wir im Röntgenspektrum messen. Der Versuchsaufbau ist in Abb.~\ref{fig:vers1}
dargestellt.
\begin{figure}[h!]
    \centering
    \includegraphics[width=0.73\textwidth]{Aufbau1.png}
    \caption{\label{fig:vers1}Skizze des Versuchsaufbaus zur Röntgenstrahlabsorption. Die Hauptbestandteile und deren 
    Funktionsweise sind im Haupttext beschrieben. Die Grafik wurde der Versuchsanleitung entnommen \cite{Anleitung}.}
\end{figure}\FloatBarrier
Die Röntgenstrahlung wird mithilfe einer Wolfram-Anode erzeugt, deren Funktionsweise 
sowie das entstehende weiße Spektrum in der vorbereitenden 
Frage \ref{subsec:FZV1} diskutiert wird. Über die Spannung $U$ zwischen Anode und 
Kathode, sowie den Strom $I$ in der Kathode kann die Energie und die Anzahl der 
Röntgenphotonen eingestellt werden. \\
Die Röntgenstrahlung fällt auf einen Einkristall (K), dessen Orientierung 
durch den Winkel $\Omega$ eingestellt werden kann. Auf einer davon unabhängigen 
Schiene befindet ein Proportionalzählrohr, das als Detektor der eintreffenden 
Röntgenphotonen dient. Die Position lässt sich über den $2\Theta$-Kreis einstellen. 
Vor dem Detektor befinden sich Blenden, die störende Untergrundsignale 
verringern. \\
Verwendet wird ein kubisch-flächenzentrierter $CaF_{2}$-Kristall mit 
einem Gitterparameter von $a= 5,463\pm 0,001\,\si{\angstrom}$, wobei 
der Strahl an der (220)-Netzebene gebeugt wird. \\
Nach richtiger Justage wir die Intensität (Anzahl der gebeugten Photonen) in 
Abhängigkeit des Einfallwinkels $\Theta$ aufgenommen. Hierzu wird der 
Einkristall und der Detektor schrittweise gedreht, wobei die Winkelschrittweite 
des Zählrohrs verdoppelt wird. Ein am Kristall gebeugtes Signal kann nur 
gemessen werden, wenn Einfalls- und Ausfallswinkel übereinstimmen, da 
sonst aufgrund unterschiedlicher Phasendifferenzen keine Interferenz beobachtbar
ist. \\ \\
Vor der eigentlichen Messung wird die Kristall- und Zählerposition justiert. 
Hierzu wird zunächst der Winkel $\Omega$ gesucht, für den der Strahl parallel
wird, was zu einem Intensitätsmaximum führt. Für erhöhte Genauigkeit wird
zunächst das Intensitätsmaximum gesucht und hiernach die Winkel, an denen die halbe 
Maximalintensität erreicht wird. Um den Detektor vor Übersättigung zu schützen, 
wird ein Metallstück in den Strahlengang eingebaut. Nachdem die optimale Stellung 
des Kristalls erreicht ist, wird mit gleichem Verfahren das Zählrohr verfahren, 
damit der Strahl mittig in den Detektorspalt fällt. Die ermittelten Winkel werden
im Steuerprogramm am Rechner eingepflegt. \\ \\
Für die Messungen stellen wir an der Röntgenröhre $I = 40\,\si{mA}$ und 
$U = 50\,\si{kV}$ ein. Zunächst wird im vorderen Winkelbereich 
$\Theta\,\in\,[5,0^{\circ},\,29,0^{\circ}]$ mit einer Schrittweite von 
$\Delta{\Theta} = 0,020^{\circ}$ gemessen, wobei die gezählten Signale 
über eine Integrationszeit von $t=1,5\,\si{s}$ aufsummiert werden. 
Im hinteren Winkelbereich $\Theta\,\in\,[22,5^{\circ},\,51,0^{\circ}]$
wird die Integrationszeit auf $t=5\,\si{s}$ erhöht. 
Diese Messung wird jeweils mit und ohne Absorberfolie (Nummer 6) durchgeführt, 
womit der erste Teil der Versuchs abgeschlossen ist. \\

\subsection{\label{subsec:vers2}Röntgenstrahlbeugung}
% MAX 