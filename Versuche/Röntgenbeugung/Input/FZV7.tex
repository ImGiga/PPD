\subsection{\label{subsec:FZV7}Peak im Pulverdiffraktogramm}
\textbf{\textit{a) Was kann man aus den Peaklagen bestimmen?}}\\
$\rightarrow$Die Auswertung von Pulverdiffraktionsmessungen startet durch die Indizierung, 
was darauf abzielt, den beobachteten Reflexen Miller'sche Indizes zuzuordnen \cite{Kristall}. 
Ziel ist es für eine gewisse Anzahl an Peaklagen (etwa zehn Stück) die Bragg-Gleichung \eqref{eq:bragg}
zu lösen. Hierbei ist die Wellenlänge durch die verwendete monochromatische Strahlung und der 
Einstrahlwinkel $\Theta$ durch die Messung gegeben. Die zu bestimmenden Variablen sind die 
Kristallstruktur und die zugehörigen Gitterparameter, welche die diskreten Werte von $d$ bestimmen.
Für hoch symmetrische Systeme ist diese Auswertung mit dem Taschenrechner möglich und schematisch in 
Ref.~\cite{Kristall} präsentiert. Für komplexere Systeme wird die Indizierung mit Computerprogrammen 
gemeistert. Insgesamt geben die Peaklagen Auskunft über das Kristallsystem und 
die Gitterparameter des untersuchten Kristallpulvers. \\

\textbf{\textit{b) Was kann man aus den Peakintensitäten bestimmen?}}\\
$\rightarrow$Durch die Peakintensitäten kann bestimmt werden, wie viele Gitterebenen(-Scharen) gleichzeitig 
Reflexion aufweisen, d.h.~den gleichen Ebenenabstand $d$ aufweisen. Hieraus können Rückschlüsse auf das 
Kristallsystem gemacht werden. Im kubischen System gibt es für manche Reflexe 48 Gitterebenen mit dem passenden Abstand, 
während es im triklinen System nur zwei gibt. Aus den relativen Peakintensitäten können Kristallsysteme daher schnell 
ausgeschlossen oder sogar identifiziert werden, solange eine zufällige Ausrichtung der Kristallite und eine ausreichend 
große Pulvermenge gegeben ist \cite{Kristall}. \\

\textbf{\textit{c) Welche Faktoren bestimmen das Profil? Was versteht man hierbei unter
axialer Divergenz?}}\\
$\rightarrow$Das Profil der Röntgenreflexe hängt von den verwendeten Instrumenten aber auch von der Probenbeschaffenheit 
ab. Die Reflexbreite ist unter Vernachlässigung anderer Störeffekte, wie Verspannungen im Kristall durch das Zermahlen,
invers proportional zur mittleren Korngröße der Kristallite. Zudem wird die Strahlverbreiterung durch Spalte, Blenden, 
Monochromatoren und sonstige optische Instrumente beeinflusst.
Die axiale Divergenz beschreibt die Ausweitung der einfallenden Strahlen einer punktförmigen Quelle 
auf das zu analysierende Pulver. 
Diese Divergenz kann mithilfe von Kollimatoren oder Blenden kontrolliert werden, 
um die Breite der Strahlen zu begrenzen, die auf das Pulver treffen. 
Eine große axiale Divergenz führt zu breiteren Reflexen im Diffraktogramm und somit zu einer 
schlechteren Auflösung \cite{Kristall}. \\

\textbf{\textit{d) Was bedeuten Atomformfaktor und Strukturfaktor anschaulich?}}\\
$\rightarrow$Zur Herleitung der betrachteten Faktoren starten wir bei Gl.~\eqref{eq:laue}. 
Ist die Laue-Bedingung erfüllt, so vereinfacht sich die Gleichung und unter Verwendung der Definition 
der Fourier-Koeffizienten $\rho_{hkl}$ ergibt sich folgender Zusammenhang (PEZ = primitive Elementarzelle) \cite{EPC}
\begin{align}
    I(\mathbf{K}) &\propto \left\vert\mathscr{A}(\mathbf{K})\right\vert^{2} \\
    \mathscr{A}(\mathbf{K}) &= \sum_{h,k,l}\rho_{hkl}\int_{V_{\text{Kristall}}}\,e^{i(\mathbf{G}_{hkl}-\mathbf{K})\cdot\mathbf{r}} \\
    \mathscr{A}(\mathbf{K} = \mathbf{G}_{hkl}) &= \frac{V_{\text{Kristall}}}{V_{\text{PEZ}}}\,\int_{V_{\text{PEZ}}}\,\rho(\mathbf{r})e^{-i \mathbf{G}_{hkl}\cdot \mathbf{r}}\,dV.
\end{align}
Teilt man nun die Ortskoordinate in eine Atom- und eine Umgebungskoordinate auf ($\mathbf{r} = \mathbf{r}_{\alpha} + \mathbf{r'}$), 
so erhalten wir
\begin{equation}
    \mathscr{A}(\mathbf{K} = \mathbf{G}_{hkl}) = \frac{V_{\text{Kristall}}}{V_{\text{PEZ}}}\,\sum_{\alpha}\,e^{-i \mathbf{G}_{hkl}\cdot \mathbf{r}_{\alpha}}\,\int_{V_{\alpha}}\,\rho_{\alpha}(\mathbf{r'})e^{-i \mathbf{G}_{hkl}\cdot \mathbf{r'}}\,dV'.
\end{equation} 
Mit dem Atomformfaktor $f_{\alpha}(\mathbf{G}_{hkl})$ als Fouriertransformation der atomaren Streudichte $\rho_{\alpha}$ ergibt 
sich auch der Strukturfaktor $\mathscr{S}_{hkl}$
\begin{equation}
    \fbox{$\mathscr{A}(\mathbf{K} = \mathbf{G}_{hkl}) = \frac{V_{\text{Kristall}}}{V_{\text{PEZ}}}\,\sum_{\alpha}\,f_{\alpha}(\mathbf{G}_{hkl})e^{-i \mathbf{G}_{hkl}\cdot \mathbf{r}_{\alpha}}
    = \frac{V_{\text{Kristall}}}{V_{\text{PEZ}}}\,\mathscr{S}_{hkl}$}.
\end{equation}
Der Atomformfaktor beschreibt die Winkelabhängigkeit der Streu-Intensität und kann mittels quantenmechanischer Methoden 
errechnet werden. Eine genauere Betrachtung ist in Ref.~\cite{Kristall} zu finden. 
Der Strukturfaktor beschreibt wie stark das Atom $\alpha$ streut und ist daher maßgeblich für die Intensitätsberechnung, die in 
der nachfolgenden Frage betrachtet wird. \\   

\textbf{\textit{e) Wie kann man die Intensität berechnen? 
Was bedeuten in diesem Zusammenhang die Begriffe Absorptions-, Lorentz-, 
Polarisations-, Extinktions- und Flächenhäuffigkeitsfaktor?}}\\
$\rightarrow$Für die gestreute Intensität gilt \cite{Kristall}
\begin{equation}
    \fbox{$I_{hkl} = I_{0}\cdot S\cdot \lambda^{3}\cdot A\cdot E\cdot Lp\cdot H\cdot \left\vert\mathscr{S}_{hkl}\right\vert^{2}$},
\end{equation}
wobei $\lambda$ und $I_{0}$, Wellenlänge und Intensität der einfallenden Welle sind. Die restlichen Größen sind Korrekturfaktoren wie 
ein Skalierungsfaktor $S$, die Absorptionskorrektur $A$, die Extinktion $E$, die Lorentz-Polarisationskorrektur $Lp$ 
und die Flächenhäufigkeit $H$ für Pulvermessungen. \\ 
Der Lorentzfaktor berücksichtigt, dass Reflexe bei Experimenten mit einem bewegten Kristall bei 
zunehmendem Streuwinkel länger in Reflexionsstellung verbleiben, was zu einer scheinbaren Intensitätszunahme führt. 
Der Polarisationsfaktor berücksichtigt die Orientierung des elektrischen Vektors der einfallenden und gestreuten Strahlung, 
da die Amplitude vom Winkel zueinander abhängt.
Die Flächenhäuffigkeit wurde bereits in Frage b) besprochen. Sie besagt, wie oft eine Gitterebenenschar mit 
gleichem Abstand $d$ vorkommt, was zu der Vervielfältigung eines Reflexes führt. 
Der Absorptionskorrekturfaktor berücksichtigt die Wechselwirkung der Strahlung mit der Materie und die damit 
einhergehenden Intensitätsverluste, die von Materialgröße und chemischer Zusammensetzung abhängen.
Der Extinktionsfaktor berücksichtigt die Mehrfachstreuung von Röntgenstrahlen und deren Interferenz, was zu 
weiteren Intensitätsunterschieden führen kann \cite{Kristall}. \\

\textbf{\textit{f) Was beschreibt der Temperaturfaktor (Debye-Waller-Faktor) und wie
wirkt er sich auf die Intensität der Reflexe aus?}}\\
$\rightarrow$Der Debye-Waller-Faktor berücksichtigt die thermische Schwingung der Atome 
um Ihre Gleichgewichtsposition und die damit einhergehende Abnahme der Streukraft. 
Zudem ist die Streuung an den ausgedehnten Atomorbitalen weniger stark für größer werdende 
Beugungswinkel. Für den Fall von isotroper Auslenkung (Atomschwingung hat keine Vorzugsrichtung), 
lässt sich die Intensitätsabnahme über den temperaturabhängigen Atomformfaktor beschreiben, 
der sich aus Gewichtung des Atomformfaktors mit dem Debye-Waller-Faktor wie folgt ergibt 
\begin{equation}
    f_{\alpha}^{T} = f_{\alpha}e^{-8\pi^{2}\langle x\rangle^{2}\frac{\sin^{2}(\Theta)}{\lambda^{2}}}. 
\end{equation}
Die Temperaturabhängigkeit kommt durch die mittlere Auslenkung $\langle x\rangle$ des Atoms ins Spiel 
und schwächt die Intensität bei sinkender Temperatur aufgrund der erhöhten thermischen Schwingung und 
der damit verbundenen Atomorbital-Ausdehnung \cite{Kristall}. \\

\textbf{\textit{g) Aus welchen Komponenten setzt sich der Untergrund in einem 
Pulverdiffraktogramm zusammen? Warum ist der Untergrund bei niedrigen Winkeln
oft viel höher?}}\\
$\rightarrow$Der Untergrund im Messsignal setzt sich aus mehreren Störsignalen zusammen. 
Zunächst ist die kontinuierliche Hintergrundstrahlung zu beachten, die durch Streuung und Reflexion 
der Strahlung in der Umgebung des Detektors verursacht wird. Diese Störsignale 
sind deutlich stärker, wenn der Einfallswinkel niedrig ist, da die Eingangsstrahlung leichter
an den Detektor gelangt. 
Zudem kann durch inkohärente Streuung der Strahlung an den Elektronen ein Untergrund entstehen. 
Außerdem ist es möglich, dass die Elektronen in den Atomen die Energie der einfallenden 
Welle absorbieren und dann durch Rückfall auf niedrigere Energielevel wieder Strahlung emittieren, 
welche ein Untergrundsignal erzeugt. 
Abschließend sind Wechselwirkungen an der Oberfläche gerade bei kleinen 
Einfallswinkeln relevant, was zu weiteren ungewünschten Signalen führt \cite{Kristall, Schwarz, EPC}.
