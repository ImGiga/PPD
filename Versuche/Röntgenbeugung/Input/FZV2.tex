\subsection{\label{subsec:FZV2}Wechselwirkung zwischen Röntgenstrahlung und Materie}
\textbf{\textit{a) Wie können Röntgenstrahlen und Materie wechselwirken?}} \\
$\rightarrow$ Beim Auftreffen von Röntgenstrahlen auf Materie können verschiedene Wechselwirkungen auftreten, insbesondere Streuung und Absorption. Absorption wird hierbei von drei verschiedenen Effekten dominiert: dem Photoeffekt, der Compton-Streuung und der Paarbildung. Streuung tritt bei Kristallen und Materialien auf, wenn die Elektronen von den Kernen im Kristall abgelenkt werden. \\
\textbf{\textit{b)Wie und wozu kann man diese jeweils in Medizin, Technik und Wissenschaft
ausnutzen?}}\\
$\rightarrow$Ein Röntgenphoton kann ein Elektron ionisieren, es also aus der Atombindung lösen. 
Dieser Vorgang wird als atomarer Photoeffekt bezeichnet und tritt vermehrt bei schweren Elementen auf, 
bedingt durch die geringere Ionisierungsenergie. 
In der medizinischen Bildgebung wird dieser Effekt genutzt, da Röntgenstrahlung von schwerem Gewebe wie Knochen absorbiert wird. 
Dadurch entsteht ein Kontrast im aufgenommenen Durchstrahlungsbild \cite{Demtroder}. \\
Für die Röntgenbildgebung mit geringer Dosis, beispielsweise in der Mammographie, 
wird die Wechselwirkung von elektromagnetischer Strahlung mit Elektronen eines Moleküls kleiner 
als die Wellenlänge der Strahlung ausgenutzt. 
Die Rayleigh-Streuung an solchen Molekülen, die quasi ohne Energieverlust der einfallenden Photonen stattfindet, 
dient dabei als Bildkontrastgeber. 
Hierfür eignet sich insbesondere die energieärmere weiche Röntgenstrahlung, 
da diese das Gewebe bei minimaler Belastung durchdringen kann \cite{Schwarz, WikiRo}. \\
Ein Röntgenphoton kann auch an einem Elektron der äußeren Schale eines Atoms inelastisch streuen, 
wobei sich die Bewegungsrichtung und die Wellenlänge (Energie) je nach Streuwinkel ändern. 
Dieser sogenannte Compton-Effekt wird in der Computertomographie genutzt, bei der der Körper mit Röntgenstrahlung durchstrahlt 
wird und die gestreuten Photonen gemessen werden. 
Die Intensität und der Winkel geben dabei Aufschluss über die Gewebeart und -dichte. 
Der Compton-Effekt findet auch Anwendung zur Fokussierung von Gammastrahlung, insbesondere in Compton-Teleskopen, 
da herkömmliche Linsen versagen. 
Weitere Anwendungsgebiete liegen in der Röntgenprüfung von industriellen Bauteilen und in der Röntgenkristallographie \cite{Schwarz, Compton}. \\
Ein weiterer Effekt ist die Paarbildung, bei der aus einem Röntgenphoton ein Elektron-Positron-Paar erzeugt wird. 
Diese Erzeugung ist je nach Energie des Photons möglich und wird durch Wechselwirkung mit einem Hüllenelektron oder dem Atomkern ausgelöst. 
Bei der Wechselwirkung mit einem Elektron entsteht zusätzlich ein freies Elektron durch die Ionisierung des Atoms. 
Der umgekehrte Prozess wird in der Bildgebung bei der Positronen-Emissions-Tomographie oder auch in der Strahlentherapie eingesetzt. \cite{WikiRo}

