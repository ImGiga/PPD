\subsection{\label{subsec:FZV9}Numerische Apertur}
\textbf{\textit{Welche Bedeutung hat die numerische Apertur NA?}} \\
$\rightarrow$
Die bereits erwähnte NA ist ein Maß dafür, wie viel Licht ein Objektiv von einem 
Objekt aufsammeln kann, dass sich im Fokus befindet. Je mehr Licht des zu untersuchenden 
Objekts aufgenommen wird, desto besser ist das Auflösungsvermögen, was direkt aus 
Antwort \ref{subsec:FZV7} folgt. Die NA wird als Produkt aus dem Öffnungswinkel $\alpha$ eines
scharf gestellten Objekts mit dem Brechungsindex $n$ des Mediums zwischen Objekt und Objektiv 
definiert. Es gilt
\begin{equation}
    \text{NA} = n\cdot\sin(\frac{\alpha}{2}).
\end{equation}
Die NA wird folglich größer, wenn $n$ steigt, weswegen im Versuch Immersionsöl zwischen 
Probe und Objektiv-Linse eingebracht ist. \\
Insgesamt ist die numerische Apertur eines Objektivs entscheidend dafür, wie 
gut das Auflösungsvermögen des Gesamtsystems ist und kann als leicht einstellbarer 
Parameter gesehen werden \cite{Auf1, Auf2, Auf3}. \newpage