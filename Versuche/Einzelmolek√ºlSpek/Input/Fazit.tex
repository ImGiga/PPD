\section{\label{sec:fazit}Fazit}
Die Einzelmolekülspektroskopie zeigt sich als effektives Verfahren, wie ihr Name verspricht. 
Im Verlauf eines komplexen Experimentes spielten sowohl die Datengewinnung als auch -verarbeitung eine zentrale Rolle. 
Wir konnten unsere eigene Probe durch Verdünnung und Verteilung mithilfe eines Spin Coaters herstellen, 
wodurch PBI in eine PS-Matrix eingebettet wurde. Dadurch war es sogar im Weitfeldmodus möglich, viele Moleküle getrennt voneinander zu beobachten. \\
Besonders beeindruckend war die Fähigkeit, das Fluoreszenzsignal einzelner Moleküle zu messen und Phänomene 
wie das Blinking und das Ausbleichen zu beobachten. Mittels eines rotierenden Polarisators zeigten wir, dass 
die Lichtabsorption effektiv nur entlang einer bevorzugten Achse erfolgt. Durch die Messung des entsprechenden 
Signals konnten nicht nur die Rotationsfrequenz des Polarisators bestimmt, sondern auch die relative Ausrichtung 
der betrachteten Moleküle zueinander erfasst werden. Abschließend ermöglichten uns zahlreiche Einzelspektren, 
auf eine Ensembleverteilung zu schließen und den statistischen Charakter einzelner Messungen im Vergleich zu Messungen im Ensemble zu verstehen. \\
Dieses Experiment war das letzte von fünf Versuchen, die wir im Rahmen unseres Masterpraktikums durchführten, und wir betrachten es als ein Highlight. 
Der gesamte Weg von der Probenpräparation bis zur eigenen Detektion einzelner Moleküle gestaltete sich äußerst spannend. 
Obwohl der Auswertungsprozess anspruchsvoll war, erweiterten wir unsere Fähigkeiten im Umgang mit Python und den entsprechenden 
Visualisierungsbibliotheken erheblich. Die einfache Handhabung des Programms ImageJ war beeindruckend und äußerst angenehm. \\
Insgesamt war dieses Experiment ein voller Erfolg, und wir sind dankbar für die Gelegenheit, daran teilgenommen zu haben.