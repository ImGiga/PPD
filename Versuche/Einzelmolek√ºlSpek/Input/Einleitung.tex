\section{\label{sec:einleitung}Einleitung}
Die Einzelmolekülspektroskopie (EMS), ein dynamisches Feld der optischen Spektroskopie, 
hat sich seit ihrem bahnbrechenden Start im Jahr 1989 rapide entwickelt. 
Durch den Nachweis einzelner Moleküle eröffnet sie neue Anwendungsbereiche 
in Wissenschaft und Technik. 
Die EMS ermöglicht es, molekulare Parameter präzise zu bestimmen, die in herkömmlichen 
Ensemble-Messungen nicht zugänglich sind, und erlaubt die Erforschung von Heterogenität.
Hierdurch können Ort des Moleküls und dessen 
Trajektorie, Einfluss der Lichtpolarisation, dynamische Zustandsänderung 
und viele weitere Effekte gemessen werden. Zudem lassen sich Quanteneffekte 
wie Photonbunching und -antibunching nachweisen, die im Ensemblemittel verborgen bleiben.
Die experimentelle Herausforderung besteht darin, das schwache Fluoreszenzsignal eines einzelnen 
Chromophors von anderen Störsignalen zu trennen, was den Einsatz hochsensibler Detektoren und 
ausgeklügelter Filtertechniken erfordert. \\
In diesem Praktikumsbericht gehen wir zunächst auf theoretische Grundlagen ein, bevor wir die 
am Praktikumstag gewonnenen Daten analysieren. 
Bei der Versuchsdurchführung wird das Chromophor Perylenbisimid (PBI) aufgrund seiner hohen 
Quantenausbeute untersucht, welches wir stark verdünnt in eine Polystyrol (PS) Matrix eingebettet. 
Hierdurch ist es uns möglich das Verhalten einzelner Moleküle zu betrachten, deren Spektren zu 
messen und deren Polarisationsabhängigkeit zu analysieren. 
Im folgenden Bericht beschreiben wir den zugrundeliegenden Versuchsaufbau, zeigen aufgenommene 
Bilder und diskutieren daraus gewonnene Daten. Abschließend werden die Ergebnisse zusammengefasst
und Schlüsse gezogen.  
