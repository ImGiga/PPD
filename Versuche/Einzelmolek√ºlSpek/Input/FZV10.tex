\subsection{\label{subsec:FZV10}Binning}
\textbf{\textit{Die CCD-Kamera wird mit einem 2×2-Binning betrieben. Welchen Vorteil bietet das Binning für das Experiment?}} \\
$\rightarrow$
Das Binning bei einer CCD-Kamera bedeutet, dass benachbarte Pixel zusammengefasst werden, 
um die Bildauflösung zu verringern. Im Fall eines 2×2-Binning wird ein Quadrat aus 4 Pixel 
zu einem einzigen Pixel zusammengefasst. Diese Technik bietet mehrere Vorteile für das Experiment. 
Zum einen erhöht dies die Empfindlichkeit, da die Signalstärke pro Detektions-Pixel vervierfacht 
wird. Aufgrund der stochastischen Natur steigt die Signalstärke des Rauschens nicht linear, 
sondern mit der Wurzel, was zu einer Verdoppelung der Rauschintensität, also einer Verdoppelung 
des Signal-Rausch-Verhältnisses führt. \\
Weitere kleine Faktoren sind die Reduzierung des Auslese-Rauschens, da weniger Auslesevorgänge 
pro Bild stattfinden, was wiederum die Bildaufnahme beschleunigt und den Speicherbedarf reduziert. 
Insgesamt führt das 2×2-Binning zu einer Verschlechterung der Bildqualität, aber einer deutlichen
Verbesserung des Signal-Rausch-Verhältnisses. Da wir in diesem Versuch einzelne Moleküle untersuchen, 
ist das Binning sinnvoll, um das schwache Fluoreszenzsignal besser detektierbar zu machen \cite{MMS}. \\

