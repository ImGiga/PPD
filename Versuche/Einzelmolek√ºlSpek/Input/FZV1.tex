\subsection{\label{subsec:FZV1}Fluoreszenz von Molekülen und Atomen}
\textbf{\textit{Was versteht man unter Fluoreszenz?}} \\
$\rightarrow$
Fluoreszenz ist eine Art von Lumineszenz, bei der ein Photon spontan von einem zuvor 
energetisch angeregten Material emittiert wird. 
Im Gegensatz zur Phosphoreszenz findet die Lichtemission ohne vorherigen    
Aktivierungsprozess statt, was daran liegt, dass der Übergang zwischen den 
energetischen Zuständen über die Auswahlregeln erlaubt ist.
Der Zeitraum zwischen der energetischen Anregung und der Photonen-Emission 
ist daher in der Regel sehr kurz. 
Die Energie des emittierten Photons ist dabei kleiner oder gleich der 
Anregungsenergie \cite{EPC, AMO, Demtroder, Prinzip}. \\

\textbf{\textit{Was ist der Unterschied zwischen den Spektren von Atomen und Molekülen?}} \\
$\rightarrow$
Die Form des Fluoreszenzspektrums eines Materials wird von vielen Faktoren beeinflusst. 
Der Haupteinfluss wird durch die besetzbaren Energiezustände und die zwischen ihnen herrschenden 
Auswahlregeln gebildet. \\
Ein Atom besitzt diskrete Energiezustände, die von den gebundenen Elektronen besetzt werden können. 
Hierbei wird die energetische Lage von der Hauptquantenzahl bzw.~der Atomschale bestimmt. 
Weitere magnetische und elektrische Wechselwirkungen können zu einer geringen energetischen 
Aufspaltung führen (Fein- und Hyperfeinstruktur). 
Ein Elektron kann nur unter Erfüllung des Energieerhaltungssatzes einen energetischen Übergang 
unter Absorption oder Emission eines Photons der Frequenz $\nu$ vollführen. Es gilt
\begin{equation}
    \Delta{E} = \left\vert E_{i} - E_{f}\right\vert = h\cdot\nu,
\end{equation} 
wobei $E_{i}$ ($E_{f}$) die Energie des Startzustandes (Endzustandes) beschreibt, und $h$ das 
Plancksche Wirkungsquantum ist. \\
Das Fluoreszenzspektrum eines Atoms besteht daher aus diskreten Linien, 
deren Lage durch die Energiedifferenz der beteiligten Zustände gegeben ist.
Die Linien besitzen aufgrund der begrenzten Lebensdauer der Zustände und der 
Energie-Zeit-Unschärferelation eine gewisse Linienbreite, die durch Atom- und 
Elektron-Wechselwirkungen vergrößert werden kann.
Die Intensität der Linie ist proportional zur spontanen Übergangswahrscheinlichkeit, 
die vom Einstein-Koeffizienten beschrieben wird. 
Dieser ist wiederum proportional zum Betragsquadrat des Übergangsdipolmoments $D_{if}$, 
das in der nachfolgenden Frage \ref{subsec:FZV2} genauer behandelt wird. \\
Zusammengefasst besteht das Fluoreszenzspektrum eines Atoms also aus scharf definierten 
Linien, die Rückschlüsse auf die diskreten Energieniveaus des Atoms und die Übergänge zwischen 
Ihnen zulassen \cite{Demtroder,EPC}. \\ \\
In Molekülen sind die besetzbaren Energieniveaus aufgrund der erhöhten Anzahl von 
Atomen und der zusätzlichen Freiheitsgrade aufgrund der räumlichen Ausdehnung 
wesentlich komplexer aufgeteilt.
Zu den elektronischen Übergängen, die wir bei Atomen beobachten, 
kommen im Molekül zusätzliche energetische Anregungen wie die Schwingungen 
der Atomkerne und die Rotationen der Molekülbestandteile hinzu. 
Diese Anregungen entsprechen ebenfalls diskreten Energiezuständen, liegen 
jedoch in einer anderen Größenordnung der Energie. 
Dies führt dazu, dass die aufgenommenen Spektren nicht mehr als diskrete Linien 
dargestellt werden, sondern als breite Banden, die einen kontinuierlichen Verlauf aufweisen. \\ 
Um ein Verständnis für die unterschiedlichen Energiebereiche der Anregungen zu entwickeln: 
Elektronische Übergänge in Atomen erstrecken sich von Elektronenvolt (eV) bis zu 
Kiloelektronenvolt (keV), während sie in Molekülen aufgrund der Elektronenwechselwirkungen 
etwas darunter im Bereich kleiner eV liegen. 
Schwingungsübergänge sind typischerweise im Bereich von Millielektronenvolt (meV) anzusiedeln, 
während Rotationsübergänge sogar im Bereich von Mikroelektronenvolt (µeV) auftreten können. 
Die diskreten Energieniveaus der zusätzlichen Freiheitsgrade verschwimmen 
aufgrund des begrenzten Auflösungsvermögens bei der Fluoreszenzspektroskopie, 
was zu einem kontinuierlichen Verlauf führt \cite{EPC, Prinzip, Parson, AMO}. \\
