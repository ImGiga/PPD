\subsection{\label{subsec:FZV5}Einzelmolekülspektroskopie}
\textbf{\textit{Was sind die Vor- und Nachteile der Einzelmolekülspektroskopie?}} \\
$\rightarrow$
Die Einzelmolekülspektroskopie bietet viele Vorteile gegenüber der Fluoreszenzmessung an 
Ensembles. Durch die Eliminierung der Ensemble-Mittelung ermöglicht sie direkte Messungen 
des Verhaltens einzelner Moleküle, was die Erkundung verborgener Heterogenität in komplexen 
Umgebungen ermöglicht. Die zeitliche Information der Messung erlaubt die direkte Beobachtung 
von Dynamik und mechanischen Zustandsänderungen sowie Zwischenschritten in chemischen 
Reaktionen. Zusätzlich können durch die Einzelmolekülspektroskopie der Einfluss der lokalen 
Umgebung auf Energielevel, Lebensdauer und Diffusion erfasst werden. Ein weiterer Vorteil 
ist die Erkennung von Spektralsprüngen aufgrund von Konfigurationsänderungen, die in 
Ensemblemessungen als zwei Peaks erscheinen würden. Darüber hinaus ermöglicht die Messung 
von Photon Antibunching Einblicke in die Quanteneigenschaften der Emissionsquelle und die 
Bestimmung von Zustandslebensdauern. \\
Die Nachteile der Einzelmolekülspektroskopie liegen hauptsächlich in den technischen Schwierigkeiten, 
insbesondere bei Messungen im Bereich von Raumtemperatur. Das gemessene Signal eines einzelnen 
Fluorophors ist sehr gering, was die Detektion erschwert und die Messdauer erhöht. 
Die Probenvorbereitung ist nicht trivial, da das Molekül oft immobilisiert sein muss, was durch 
Einbindung in Polymere, hochviskose Flüssigkeiten oder Anheften an Oberflächen erreicht werden kann. 
Im Vergleich zur Ensemblemessung sind viele Messungen erforderlich, um statistische Aussagekraft 
zu erreichen, was zu einem deutlich aufwändigeren Messprozess führt. Zudem müssen Verunreinigungen 
in der Probe, optische Komponentenemissionen und Dunkelstromzählungen des Detektors berücksichtigt 
werden. Die inelastische Ramanstreuung von Licht kann ebenfalls auftreten und erfordert die Auswahl 
sehr kleiner untersuchter Lösungsmittelvolumina, um das Fluoreszenzsignal vom Streulichtsignal 
zu unterscheiden. Schließlich können einzelne Moleküle durch chemische Reaktionen ihre 
Emissionsfähigkeit verlieren, was weitere Messungen an ihnen unmöglich macht. \\
Trotz dieser Herausforderungen hat die Einzelmolekülspektroskopie aufgrund der Vielzahl an neuen 
Informationen, die gemessen werden können, das Interesse an experimenteller Forschung in diesem 
Bereich stark belebt \cite{EPC, P1, P2, SingEns, Prinzip}. \\