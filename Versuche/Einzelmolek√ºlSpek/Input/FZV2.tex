\subsection{\label{subsec:FZV2}Franck-Condon Prinzip}
\textbf{\textit{Erläutern Sie das Franck-Condon Prinzip.}} \\
$\rightarrow$
Im Folgenden leiten wir den Franck-Condon Faktor und das damit verbundene Prinzip 
schematisch her. \\ 
Sieht man das einfallende elektromagnetische Feld des Anregungslichtes als 
Störung, so kann man mithilfe des Dipol-Operators $\hat{\mu}$ und 
Fermis Goldener Regel die Übergangswahrscheinlichkeit eines elektronischen 
Übergangs vom Zustand $i$ nach $f$ bestimmen. \newpage
Hierfür wird das bereits 
in vorheriger Frage erwähnte Übergangsdipolmoment $D_{if}$ benötigt, welches sich 
folgendermaßen ergibt 
\begin{equation}\label{eq:deflangle}
    D_{if} = \langle\Psi_{i}\vert \hat{\mu} \vert \Psi_{f}\rangle = \int\int\,d\mathbf{r}^{N_{\text{Elektron}}}\,d\mathbf{R}^{N_{\text{Kern}}} \Psi_{i}^{*}\hat{\mu}\Psi_{f},
\end{equation}
wobei $\Psi$ die von $\mathbf{R}$ und $\mathbf{r}$ also Kern- und Elektronenkoordinaten abhängigen Gesamtwellenfunktionen 
der jeweiligen Zustände sind. \\
Nach der Born-Oppenheimer Näherung kann die Zustandswellenfunktion des Moleküls in einen Anteil der 
Elektronen und einen Anteil der Kerne wie folgt aufgeteilt werden
\begin{equation}
    \Psi_{\text{Molekül}}(\mathbf{r}^{N_{\text{Elektron}}}, \mathbf{R}^{N_{\text{Kern}}}) \approx \chi_{\text{Kern}}(\mathbf{R}^{N_{\text{Kern}}})\,\phi_{\text{Elektron}}(\mathbf{r}^{N_{\text{Elektron}}}, \mathbf{R}^{N_{\text{Kern}}}).
\end{equation}
Diese Näherung ist möglich und sinnvoll, da die Atomkerne sehr viel schwerer als die Elektronen sind und daher quasi stillstehen. \\
Für die Herleitung des Franck-Condon Faktors wird diese Argumentation übernommen und man nimmt an, dass die elektronischen 
Übergänge sehr viel schneller stattfinden als die Kerne der Bewegung folgen können. Der Übergang wird folglich 
in zwei Schritte aufgeteilt, wonach sich im ersten Schritt die Elektronenwellenfunktion instantan ändert 
und hiernach die Kernposition angepasst wird. Für die Berechnung folgt daher, dass die Kernposition 
$\mathbf{R}$ in der Elektronenwellenfunktion nur mehr ein Parameter und keine Variable mehr ist. \\
Teilt man zusätzlich den Dipol-Operator durch die Position der positiven und negativen Ladungsträger auf, 
so folgt 
\begin{align}
    \hat{\mu} &= \hat{\mu}_{\text{E}} + \hat{\mu}_{\text{K}} \\
    &= -e\mathbf{r}^{N_{\text{E}}} + q_{\text{K}}\mathbf{R}^{N_{\text{K}}} \\
    \Rightarrow D_{if} &= \langle\chi_{i}\psi_{i}\vert \hat{\mu}_{\text{E}} \vert\chi_{f}\psi_{f}\rangle + 
    \langle\chi_{i}\psi_{i}\vert \hat{\mu}_{\text{K}} \vert\chi_{f}\psi_{f}\rangle \\
    &= \langle\chi_{i}\vert \chi_{f}\rangle \langle \psi_{i}\vert \hat{\mu}_{\text{E}}\vert\psi_{f}\rangle.
\end{align} 
Im letzten Schritt wurden die Integrale aufgeteilt, da $\mathbf{R}$ nur ein Parameter der Elektronenwellenfunktion 
darstellt und der Beitrag über das Kern-Dipolmoment verschwindet, da die Elektronenwellenfunktionen
orthogonal zueinander stehen und damit das Integral zwischen zwei unterschiedlichen Zuständen verschwindet. \\
Der Resultierende Vorfaktor $\langle\chi_{i}\vert \chi_{f}\rangle$ wird Franck-Condon Faktor genannt, 
wobei dessen Betragsquadrat proportional zur Übergangswahrscheinlichkeit und damit zur Strahlungsintensität 
des betrachteten Übergangs ist. \\
Nach Gl.~\eqref{eq:deflangle} gibt der Faktor Auskunft über die räumliche Überlappung zweier Kern-Schwingungs-Wellenfunktionen
und besagt, dass der Übergang für gute Überlagerung wahrscheinlicher ist. \\ 
Zusammengefasst beschreibt das Franck-Condon Prinzip den Umstand, dass elektronische (schwingungs-)Übergänge 
wahrscheinlicher sind, wenn die Wellenfunktionen der Kerne räumlich stärker überlappen. Das Prinzip erscheint sinnvoll, 
da die Elektronenbewegung deutlich schneller abläuft und daher ein senkrechter Übergang nur stattfinden kann, 
wenn sich am betrachteten Ort ein Atomkern befinden kann. Diese Aufenthaltswahrscheinlichkeit wird über den 
Franck-Condon Faktor berechnet \cite{EPC,Demtroder,Parson}. \\ 