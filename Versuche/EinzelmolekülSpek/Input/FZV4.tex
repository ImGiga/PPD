\subsection{\label{subsec:FZV4}Einzel- und Ensemblemessung}
\textbf{\textit{Was unterscheidet das Fluoreszenzsignal eines einzelnen Moleküls von dem eines Ensembles?}} \\
$\rightarrow$
Zuerst betrachten wir das Fluoreszenzsignal eines einzelnen Moleküls und gehen dann zur Ensemblemessung über.
Das Signal, das von einem Molekül erzeugt wird, ist stark abhängig von seiner Konfiguration und der Polarisation
des anregenden Lichts. Manche Konfigurationen führen dazu, dass das Molekül entweder keine Photonen emittiert
oder nur einen Teil der maximalen Fluoreszenzintensität zeigt. Zudem können Moleküle unterschiedlich stark von
Licht mit unterschiedlicher Polarisation angeregt werden, was zu einer polarisationsabhängigen Intensität führt. \\
Bei einer Ensemblemessung werden die Fluoreszenzsignale vieler betrachteter Moleküle gemittelt. Dabei gehen
Informationen über Konfigurationsunterschiede und in isotropen Proben auch die Polarisationseinflüsse verloren.
Das Signal-Rausch-Verhältnis ist bei der Ensemblemessung deutlich besser, da die Quantenausbeute vieler Fluorophore
die Signalintensität bestimmt. Konstanter Untergrund und andere Störeffekte spielen eine entscheidendere Rolle,
wenn das Signal eines einzelnen Moleküls detektiert wird. \\
Des Weiteren können bei der Messung des Fluoreszenzsignals einzelner Moleküle Effekte wie Photobleaching oder
Blinking beobachtet werden, die bei Ensemblemessungen nicht detektierbar sind, da sie durch die Mittelung
wegfallen. Zudem ist es möglich quantenmechanische Effekte wie Photonbunching oder Antibunching zu beobachten,
da von einem Fluorophor in einem bestimmten Zeitintervall nur ein Photon emittiert werden kann. \\
Da es durch Wechselwirkungen der Moleküle untereinander und mit der Umgebung zu energetischen
Verschiebungen kommen kann, ist das Fluoreszenzsignal eines Ensembles breiter und weniger scharf als das
eines einzelnen Moleküls, da sich die einzelnen verschobenen Spektren durch Mittelung überlagern. \\
Durch Detektion einzelner Moleküle ist zudem eine Orts- und Zeitauflösung möglich, die bei Ensemblemessungen
nicht erreicht werden kann. \\
Zusammenfassend ist das gemessene Fluoreszenzsignal eines einzelnen Moleküls intensitätsärmer und verrauschter,
es enthält Informationen über Heterogenitäten und zeigt andere Linienformen mit geringerer Breite,
was die Auflösung nah beieinander liegender Fluoreszenzlinien ermöglicht \cite{P1, P2, Demtroder, SingEns}. \\
