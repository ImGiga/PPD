\subsection{\label{subsec:FZV4}Einzel- und Ensemblemessung}
\textbf{\textit{Was unterscheidet das Fluoreszenzsignal eines einzelnen Moleküls von dem eines Ensembles?}} \\
$\rightarrow$
Zuerst betrachten wir das Fluoreszenzsignal eines einzelnen Moleküls und gehen dann zur Ensemblemessung über. 
Das Signal, das von einem Molekül erzeugt wird, ist stark abhängig von seiner Konfiguration und der Polarisation 
des anregenden Lichts. Manche Konfigurationen führen dazu, dass das Molekül entweder keine Photonen emittiert 
oder nur einen Teil der maximalen Fluoreszenzintensität zeigt. Zudem können Moleküle unterschiedlich stark von 
Licht mit unterschiedlicher Polarisation angeregt werden, was zu einer polarisationsabhängigen Intensität führt. \\
Bei einer Ensemblemessung werden die Fluoreszenzsignale vieler betrachteter Moleküle gemittelt. Dabei gehen 
Informationen über Konfigurationsunterschiede und in isotropen Proben auch die Polarisationseinflüsse verloren. 
Das Signal-Rausch-Verhältnis ist bei der Ensemblemessung deutlich besser, da die Quantenausbeute vieler Fluorophore 
die Signalintensität bestimmt. Konstanter Untergrund und andere Störeffekte spielen eine entscheidendere Rolle, 
wenn das Signal eines einzelnen Moleküls detektiert wird. \\
Des Weiteren kommt es zu einer Veränderung der detektierten Linienform und einer damit einhergehenden Veränderung 
der Auflösung zweier energetisch naher Übergänge. In der Einzelmolekülspektroskopie wird das betrachtete Molekül 
beinahe unbeweglich gehalten, wodurch es nicht zu einer Linienverbreiterung durch Stoßprozesse oder den Dopplereffekt 
kommt. Die Spektrallinie hat daher die Form eines Lorentz-Profils, dessen Breite der natürlichen Linienbreite entspricht, 
die nur durch die Lebensdauer des angeregten Zustands bestimmt wird. \\
Für Ensemblemessungen müssen die Effekte der Linienverbreiterung berücksichtigt werden, was zu einer Veränderung der 
Linienform hin zu einem Gauß-Profil führt. Zudem führen Wechselwirkungen zwischen den Molekülen in der Ensemble-Probe 
zu einer Verschiebung der energetischen Niveaus und einer zusätzlichen Verschmierung des Fluoreszenzsignals. \\
Zusammenfassend ist das gemessene Fluoreszenzsignal eines einzelnen Moleküls intensitätsärmer und verrauschter, es enthält 
Informationen über Heterogenitäten und zeigt andere Linienformen mit geringerer Breite, was die Auflösung nah beieinander 
liegender Fluoreszenzlinien ermöglicht \cite{P1, P2, Demtroder, SingEns}. \\
