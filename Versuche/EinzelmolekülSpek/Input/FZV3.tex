\subsection{\label{subsec:FZV3}Quantenausbeute}
\textbf{\textit{Was versteht man unter Quantenausbeute?}} \\
$\rightarrow$
Die Quantenausbeute bezieht sich auf die Effizienz, mit der ein System Photonen oder Elektronen absorbieren,
emittieren oder in einem bestimmten Prozess nutzen kann.
Die Quantenausbeute wird oft als Verhältnis der Anzahl der gewünschten Ereignisse
(z.B.~emittierte Photonen, produzierte Elektronen) zur Anzahl der eingesetzten Quellen
(z.B.~eingebrachte Energie, eingestrahlte Photonen) definiert. \\
In der Fluoreszenzspektroskopie bezeichnet die Quantenausbeute den Anteil der Moleküle,
die ein Photon emittieren, an der Gesamtanzahl der Moleküle, die energetisch angeregt wurden \cite{Prinzip,Demtroder}. \\
% Will man die Anzahl an Fluoreszenzphotonen $N_{\text{F}}$ berechnen, die auf dem Detektor im Raumwinkel $\Omega$ 
% auftreffen, so benötigt man die Anzahl der eingestrahlten Photonen $N_{\text{in}}$, die 
% Absorptionswahrscheinlichkeit $\kappa$ und die Quantenausbeute \cite{Demtroder}
% \begin{equation}
%     N_{\text{F}} = N_{\text{in}}\cdot \kappa \cdot \epsilon \cdot \frac{\Omega}{4\pi}.
% \end{equation}
% Die Quantenausbeute kann über gemessene Standardwerte und Intensitätsverhältnisse errechnet werden, 
% worauf wir hier nicht weiter eingehen und auf Ref.~\cite{Prinzip} verweisen. \\