\subsection{\label{subsec:FZV8}Point-spread-function}
\textbf{\textit{Was ist eine point-spread-function (PSF)?}} \\
$\rightarrow$
Die PSF beschreibt die Auswirkung eines abbildenden Systems auf das Signal einer 
punktförmigen Quelle. Ist die PSF eines optischen Systems bekannt, so kann das 
entstehende Bild durch Faltung dieser Funktion mit der Abmessung des untersuchten 
Objekts, errechnet werden. Die in Antwort \ref{subsec:FZV7} erwähnten Beugungsscheibchen
entsprechen daher der PSF einer kreisförmigen Blende, da die Faltung mit einem Punkt einer 
Selbstabbildung gleicht. Folglich kann mit der PSF das Auflösungsvermögen eines optischen Systems 
berechnet werden, welches sich von einem klassischen Mikroskopieaufbau unterscheidet und 
daher Gl.~\eqref{eq:aufl} nicht folgt. 
Die Funktion selbst kann über Beugungsintegrale (z.B.~Fresnel-Kirchhoffsches Beugungsintegral)
bestimmt oder direkt mithilfe kleiner fluoreszierender Kügelchen gemessen werden \cite{Auf1, Auf2, Auf3}. \\ 
