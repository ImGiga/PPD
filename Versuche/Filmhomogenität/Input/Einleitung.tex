\section{\label{sec:einleitung}Einleitung}
In diesem Versuch wird die Produktion und Charakterisierung von dünnen Filmen mithilfe von Weißlichtreflektometrie untersucht. Die Herstellung und Kenntnis von solch dünnen Filmen ist in vielen Bereichen nicht mehr wegzudenken, beispielsweise in der Mikroelektronik.
Eine Methode zur Erzeugung solcher Filme ist Spincoating, womit man im bisherigen Studium jedoch wenig bis gar nichts zu tun hatte. Um dies zu ändern, liegt ein wesentlicher Fokus dieses Praktikums auf der Probenpräparation im Labor. Dort wird eine Polystyrollösung auf ein Substrat gespinnt und anschließend die Abhängigkeit der Filmdicke sowohl von der Konzentration als auch von der Rotationsgeschwindigkeit untersucht.
Ziel des Versuchs ist es 