\subsection{\label{subsec:FZV5}Substrat}
Ein Substrat fungiert als feste Unterlage, auf der andere Materialien angebracht werden können. 
Der Halbleiter kann selbst als Substrat fungieren, welcher unter Komposition 
anderer Halbleiter zu elektronischen Komponenten (Dioden, Transistoren, etc.) oder Chips weiter verarbeitet wird. 
Außerdem können viele organische Halbleiter zu dünnen Filmen verarbeitet werden, die dann mit vielen Substraten 
(Folien, Metalle, Glas und Papier) kompatibel sind.  
Gängige Halbleiter-Substrate sind vor allem Silizium (Si) und Gallium (Ga), aber auch Verbindungen wie 
Siliziumcarbid (SiC), Galliumarsenid (GaAs), Galliumnitrid (GaN), Indiumphosphid (InP) \cite{Substrat, EPC}. 