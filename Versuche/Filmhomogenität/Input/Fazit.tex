\section{\label{sec:fazit}Zusammenfassung}
In diesem Versuch wurde die Homogenität von dünnen Filmen untersucht und die Weißlichtreflektometrie kennengelernt. Dazu wurde eine Polystyrollösung auf einem Substrat gespincoatet und anschließend die Abhängigkeit der Filmdicke sowohl von der Konzentration als auch von der Rotationsgeschwindigkeit qualitativ und quanititav untersucht. Zudem konnte das Polystyrol einem Modell zugeordnet werden. Ein besonderer Fokus bei diesem Versuch lag auf der Probenpräparation im Labor, was insbesondere für den Theoretiker der Gruppe, aber auch allgemein etwas Unbekanntes und eine willkommene Abwechslung zum ewigen Laser justieren war. Gerade deshalb und wegen der lustigen Betreuung hat uns dieser Versuch viel Spaß gemacht und wir konnten viel Neues lernen. 