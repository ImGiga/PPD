\subsection{\label{subsec:FZV5}Spektrale Auflösung}
\textbf{\textit{Wie verändert eine kleinere Schrittweite die spektrale Auflösung einer THz-TDS Messung?}}\\
$\rightarrow$Um eine beliebig genaue Rekonstruktion des Signals zu sichern 
müssen nach dem Nyquist-Shannon-Abtasttheorem die auftauchenden Signalfrequenzen 
unterhalb einer gewissen Grenzfrequenz $f_{\text{Ny}}$ liegen, die durch die 
Abtastfrequenz $f_\text{sample}$ wie folgt beeinflusst wird
\begin{equation}
    f_{\text{Ny}} = \frac{1}{2}f_\text{sample}.
\end{equation}
Die Abtastfrequenz wird bei unserer Messung von der Motorschrittweite $\Delta l$ des variablen 
Interferrometerarms vorgegeben. Es gilt
\begin{equation}
    f_\text{sample} = \frac{1}{\Delta t} = \frac{c}{\Delta s} = \frac{c}{2\Delta l},
\end{equation}
woraus folgt, dass eine kleinere Schrittweite die spektrale Auflösung der THz-TDSMessung
verbessert. \\

\textbf{\textit{Was versteht man unter \glqq Zero-Padding\grqq{}?}}\\
$\rightarrow$Da wir mit diskreten Messungen arbeiten, werden die Signale über eine
diskrete Fourier-Transformation (DFT) in den reziproken Raum transformiert. 
Beim Zero-Padding werden schlichtweg Nullen am Ende eines zeitabhängigen Signals ergänzt, 
um dessen Länge zu erweitern. Zum einen kann das Signal dadurch auf eine Zweierpotenz-Anzahl 
von Messdaten ergänzt werden, was die Anwendungen eines schnellen DFT-Algorithmusses, der 
FFT (fast Fourier transform), erlaubt. Zum anderen erhöht sich damit die Anzahl an diskreten Punkten, 
die der Anzahl an Auflösungspunkten im Fourierraum entsprechen.
Hierdurch lassen sich eng liegende Signalfrequenzen besser voneinander lösgelöst darstellen und 
Frequenz-Artefakte reduzieren. 

