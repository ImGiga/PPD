\subsection{\label{sec:A21}Referenzmessung}
Gemäß Abschnitt \ref{sec:aufbau2} wird zunächst die Fokusgröße mithilfe einer Irisblende abgeschätzt. 
Die einstellbaren Parameter werden dabei wie folgt festgelegt:
\begin{equation}
    \Delta s = 0,005\,\si{mm}, \hspace{1.5cm} N_{\text{s}} = 100, \hspace{1.5cm} N_{\text{sample}} = 2, \hspace{1.5cm} s_{0} = 152,7\,\si{mm}. 
\end{equation}
Dies ermöglicht eine schnelle Übersichtsmessung. 
Nach der korrekten Positionierung der Irisblende im Fokus der Linse wird die Blende 
so weit geschlossen, bis die Amplitude des THz-Pulses beginnt abzunehmen. 
Die Größe wird dann mithilfe eines Messschiebers bestimmt, was zu folgendem Ergebnis führt
\begin{equation}
    \fbox{$d_{\text{fokus}} = (1.8 \pm 0.2)\,\si{mm}$}.
\end{equation}
Der Fehler wird hierbei abgeschätzt und resultiert hauptsächlich 
aus Unsicherheiten bei der Fokusposition sowie bei der Bestimmung der Blendenweite 
mittels des Messschiebers und der gescannten Signalamplitude. \\
Die Fokusgröße, wie in der vorbereitenden Frage \ref{subsec:FZV3} erwähnt, 
ist eine frequenzabhängige Größe. 
Es zeigt sich, dass der gemessene Wert eine Abschätzung für den niederfrequenten
Anteil der THz-Strahlung darstellt, da er im Bereich des theoretisch erwarteten Wertes 
für eine Strahlungsfrequenz von $0,5\,\si{THz}$ liegt. \\
Anschließend wird die Referenzmessung durchgeführt, die den THz-Puls als Funktion 
der Verzögerung aufzeichnet, ohne dass dieser eine Probe durchläuft. 
Die Messparameter werden dabei wie folgt gewählt
\begin{equation}
    \Delta s = 0,001\,\si{mm}, \hspace{1.5cm} N_{\text{s}} = 2500, \hspace{1.5cm} N_{\text{sample}} = 100, \hspace{1.5cm} s_{0} = 152,7\,\si{mm}, 
\end{equation}
was in einer schön aufgelösten Referenzmessung resultiert, die in den folgenden 
Abschnitten betrachtet wird. \\