\subsection{\label{subsec:FZV6}Zeitliche Verzögerung}
\textbf{\textit{Die Ausbreitungsgeschwindigkeit des THz-Pulses in Silizium ist langsamer als in
Luft. D.h. der THz-Puls wird in einer THz-TDS-Messung später aufgezeichnet,
wenn er durch 380 $\mathbf{\mu}$m dickes Silizium propagiert. Berechnen Sie diese zeitliche
Verzögerung. Welchen Brechungsindex müssen Sie für Silizium annehmen?}}\\
$\rightarrow$Da sich der THz-Puls aus mehreren Frequenzkomponenten zusammensetzt, wird 
die Ausbreitungsgeschwindigkeit nicht über die Phasengeschwindigkeit (Bewegung der Phasenfront)
sondern über die Gruppengeschwindigkeit (Energie- und Informationstransport) definiert,
für welche gilt
\begin{align}
    v_{\text{g}} &= \frac{d\omega}{dk} = \frac{d(ck/n(k))}{dk} \\
    &= \frac{c}{n(k_{0})} - \frac{ck}{n(k_{0})^{2}}\frac{\partial n}{\partial k} \\
    &= \frac{c}{n(k_{0})}\left(1 - \frac{k}{n(k_{0})}\frac{\partial n}{\partial k}\right).
\end{align}
Innerhalb des betrachteten THz-Bereichs ändert sich der Brechungsindex von Silizium nahezu nicht und kann 
daher als konstant angenommen werden ($\frac{\partial n}{\partial k}\approx0$) \cite{Q9}.
Für unsere Berechnungen nehmen wir $n_{\text{Si}} = \sqrt{\epsilon_{\infty}} = 3,4175$ an \cite{Anleitung}.
Die zeitliche Verzögerung des Pulses berechnet sich daher wie folgt
\begin{align}
    \Delta t &= t_{\text{Si}} - t_{\text{Luft}} \\
    &= \frac{d}{c}(n_{\text{Si}}-n_{\text{Luft}}) \label{eq:a22}\\
    \Rightarrow \Aboxed{\Delta t & \approx 3,064\,\si{ps}}.
\end{align}
