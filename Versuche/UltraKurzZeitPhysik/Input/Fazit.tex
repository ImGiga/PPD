\section{\label{sec:fazit}Fazit}
Während des Experimentiertages konnten wir einen tiefen Einblick in die Versuchsaufbauten und Funktionsweisen 
der Ultrakurzzeit-Physik gewinnen. Im ersten Abschnitt haben wir uns intensiv mit der Femtosekunden-Autokorrelation 
beschäftigt, was uns ermöglichte, den Umgang mit optischen Instrumenten zu erlernen, diese korrekt einzustellen und 
die Herausforderungen der Experimentalphysik zu bewältigen. Durch die vorbereitenden Fragen konnten wir ein umfassenderes 
Verständnis für die Aufgaben und Probleme dieses Versuchsbereichs entwickeln. Die Auswertung gestaltete sich angenehm 
und gut nachvollziehbar, da alle notwendigen Schritte klar erklärt wurden und die einführenden Kapitel einen umfangreichen 
Überblick boten. \\
Im zweiten Teil des Experiments, in dem wir THz-TDS anwendeten, vertiefte sich unser Verständnis für ein bereits theoretisch 
erlerntes Thema. Es ist stets bereichernd, theoretische Konzepte aus der Vorlesung in der Praxis anzuwenden, um neue 
Verbindungen herzustellen und das Gesamtkonzept besser zu begreifen. Die Auswertung stellte sich als anspruchsvoll heraus, 
insbesondere bei der Berechnung der spektralen Amplitude, die eine eigene Wissenschaft darstellt. Hier wäre es sinnvoll, 
ein erklärendes Kapitel oder vorbereitende Fragen in die Versuchsanleitung aufzunehmen. Die Verwendung eines bereitgestellten 
Programms für die Berechnung trägt nicht zum Verständnis bei, da die zugrunde liegende Vorgehensweise nicht erläutert wird. \\
Insgesamt können wir diesen ersten Versuch im Rahmen des physikalischen Praktikums für Fortgeschrittene als erfolgreich verbuchen. 
Die Experimentaldurchführung war faszinierend und bereitete Freude, die Ergebnisse sind vernünftig und scheinen im erwarteten 
Bereich zu liegen. Durch die Auswertung konnten wir viele neue Erkenntnisse gewinnen, die in Zukunft sicherlich wieder von Nutzen 
sein werden.