\subsection{\label{subsec:FZV4}Frage 4}
\textbf{\textit{Informieren Sie sich über die fundamentalen Eigenschaften anorganischer Halbleiter.
Was ist der Unterschied zwischen HR-Silizium (high resistivity bzw. hochohmiges
Silizium) und p-dotiertem Silizium?}}\\
$\rightarrow$Anorganische Halbleiter sind Materialien, die elektrische Leitfähigkeit zwischen Leitern 
(Metallen) und Nichtleitern (Isolatoren) aufweisen. 
Die grundlegenden Eigenschaften dieser Materialien umfassen ihre Kristallstruktur, 
Elektronenbandstruktur, Dotierungsmöglichkeiten und die Art der Ladungsträgerbewegung.
Die Fermienergie liegt für Halbleiter zwischen dem bei $T=0\,\si{K}$ vollbesetzten 
Valenzband und dem leeren Leitungsband, wobei die Bandlücke kleiner als bei Isolatoren ist, 
was die Anregung von Elektronen in das Leitungsband durch Energiezufuhr ermöglicht. 
Man unterscheidet zudem intrinsiche und dotierte Halbleiter. 
Bei intrinsischen Halbleitern wie reinem Silizium ist die Ladungsträgerkonzentration von 
gleichermaßen Elektronen und Löchern gegeben, die durch thermische Anregung erzeugt wird. 
Dotierte Halbleiter hingegen werden gezielt mit Fremdatomen im Kristallgitter verunreinigt, 
was zu neuen Energieniveaus führt, die die Ladungsträgeranregung energetische begünstigen. 
Hierdurch wird die Leitfähigkeit des Halbleiters erheblich verbessert. \\
HR-Silizium (hochohmiges Silizium) und p-dotiertes Silizium sind zwei verschiedene Arten von Silizium, 
die für unterschiedliche Anwendungen in der Halbleiterindustrie eingesetzt werden.
Das HR-Silizium weist eine sehr hohe elektrische Resistivität auf, die eine geringe Leitfähigkeit 
zur Folge hat. Anwendungen sind hochpräzise Widerstände, 
Isolatoren, Sensoren und andere elektronische Bauteile, bei denen eine geringe 
elektrische Beeinflussung oder minimale Streuung von elektrischen Signalen erforderlich ist.
Zur Erreichung des hohen speziefischen Widerstandes wird Intrinsisches Silizium mit 
möglichst wenig vorkommenden Verunreinigungen gezüchtet. Zudem werden unvermeidbaren Verunreinigungen 
durch gegensätzliche Dotierung zugefügt, die zur Kompensation führen und somit die Ladungsträgerkonzentration
weiter veringern. \\
P-dotiertes Silizium wird gezielt mit sog.~Akzeptoratomen (z.B.~Bor, Aluminium oder Gallium)
verunreinigt, die eine geringere Wertigkeit 
besitzen. Hierdurch entstehen Energieniveaus (direkt über der Valenzbandkante) 
die leichter von den Elektronen aus dem Valenzband besetzt werden können, wodurch Löcher im 
Band entstehen. Diese Löcher sind beweglich und erhöhen somit die positive 
Ladungsträgerkonzentration im Siliziumkristallgitter. 
P-dotiertes Silizium wird in der Herstellung von p-Typ-Transistoren, 
Dioden und anderen Halbleiterbauelementen verwendet und spielt somit 
eine wichtige Rolle in der Schaffung von logischen Schaltungen 
und elektronischen Schaltkreisen. \\
Der wesentliche Unterschied liegt in den elektrischen Eigenschaften und Anwendungen. 
HR-Silizium zeichnet sich durch hohe Resistivität aus und wird in Anwendungen mit minimaler 
elektrischer Leitfähigkeit eingesetzt. 
P-dotiertes Silizium hingegen zeigt erhöhte Leitfähigkeit durch Dotierung 
mit Akzeptor-Atomen und wird in der Produktion von Halbleiterbauelementen verwendet, 
die positive Ladungsträger erfordern \cite{Demtroder, EPC}. \\ 

\textbf{\textit{Erwarten Sie einen Unterschied in der THz-Transmission?}}\\
$\rightarrow$Die THz-Strahlung kann mit den freien Ladungsträgern wechselwirken oder
gebundene Ladungsträger in das Leitungsband anregen und absorbiert werden. 
Sind mehr Energieniveaus vorhanden, können Photonen unterschiedlicher Energie besser
absorbiert werden und gleichzeitig führt die Dotierung zum allgemeinen 
Anstieg freier Ladungsträger, die widerum mit der Strahlung wechselwirken können.
Insgesamt erwarten wir daher eine geringe Transmission des p-dotierten im
Vergleich zum hochohmigen Silizium. 