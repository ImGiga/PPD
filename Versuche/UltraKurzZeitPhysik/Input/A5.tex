\subsection{\label{sec:A24}Metallischer Lack}
Abschließen führen wir noch THz-TDS-Messungen an mit Nagellack beschichteten Frischhaltefolien 
durch. Hierzu wird zunächst eine unbeschichtete Folie als Referenz-Probe aufgenommen, um die 
Messungen mit den verschiedenen Lackbeschichtungen vergleichbar zu machen. Die Messparameter 
sind für alle Proben identisch und wie folgt gewählt
\begin{equation}
    \Delta s = 0,005\,\si{mm}, \hspace{1.5cm} N_{\text{s}} = 150, \hspace{1.5cm} N_{\text{sample}} = 100, \hspace{1.5cm} s_{0} = 152,7\,\si{mm}.
\end{equation} 
Die reduzierte Schrittweite in Kombination mit der verringerten Schrittzahl, erlaubt es uns schnell den 
Hauptpuls durch die Probe aufzunehmen. \\
Insgesamt vergleichen wir vier Proben mit der Referenzmessung, was vergleichend in Abb.~\ref{fig:lack} 
präsentiert wird. Zunächst werden ein roter und ein magnetischer Nagellack, sowie ein Silberleitlack 
in gleicher Dicke auf die Frischhaltefolie aufgetragen, welche in einen Probenhalter eingespannt wird. 
Um zusätzlich den Einfluss der Dicke zu untersuchen, wird eine weitere Probe unter Verwendung 
des magnetischen Nagellacks hergestellt, die großzügiger aufgetragen wird.
\begin{figure}[h!]
    \centering
    \includegraphics[trim={1.8cm 0.5cm 3.2cm 2.5cm}, clip, width=\textwidth]{Lack.pdf}
    \caption{\label{fig:lack}Das zum E-Feld des THz-Pulses gemessene Spannungssignal als Funktion der 
    zeitlichen Verzögerung $\Delta\tau$ für verschieden beschichtete Proben. Zusätzlich sind vertikale 
    Linien ausgehend von den jeweiligen Amplituden eingezeichnet, um die zeitliche Verzögerung 
    zwischen den Signalen besser abschätzen zu können. Das Darstellungsformat ist bewusst 
    verzerrt gewählt, um die Übersichtlichkeit zu verbessern.}
\end{figure}\FloatBarrier
Zunächst ist zu erkennen, dass jede Probe einen 
zeitlichen Versatz zum Referenzpuls aufweist. 
Dies ist zu erwarten, da der Lack eine zusätzliche 
Schicht darstellt, die einen realen Brechungsindex
unterschiedlich als der von Luft besitzt. Hierdurch ergibt sich
ein zeitlicher Versatz, der in der 
vorbereitenden Frage \ref{subsec:FZV6} beispielhaft
durchgerechnet ist. Da der Versatz proportional zum 
Produkt aus Dicke und Wert des Brechungsindex ist, 
kann man bei gleicher Beschichtungsdicke einen 
Unterschied im Brechungsindex ableiten. 
Für die im Experiment erstellten Proben kann 
jedoch keine einheitliche Schichtdicke garantiert werden.
Erwartungsgemäß ist der Puls, der durch die dicke magnetische Schicht propagiert
am weitesten versetzt. Der Puls, der durch den Silberleitlack 
transmittiert ist, weist innerhalb des Messfehlers nahezu keinen 
Versatz auf. Dies kann darauf zurückzuführen sein, 
dass bei der Beschichtung zu wenig Silberpartikel aufgetragen 
wurde und die Hauptkomponente durch das Lösungsmittel 
im Leitlack dargestellt wird. 
Der Hauptbestandteil 
des magnetischen Nagellacks, Eisen, hat im betrachteten Wellenlängenbereich 
einen deutlich größeren Brechungsindex als Luft, was zu einer Verzögerung des 
Pulses führt \cite{Database}. \\
Zudem sind die Signalamplituden unterschiedlich stark, was 
auf die unterschiedliche Absorption der THz-Strahlung zurückzuführen 
ist. Der rote Lack absorbiert wenig Strahlung, da es sich 
um einen nicht leitenden Stoff handelt. Daher ist eine 
große Bandlücke zu vermuten, zu deren Überwindung die 
Strahlungsenergie nicht ausreicht und eine Wechselwirkung mit freien 
Ladungsträgern aufgrund deren geringer Konzentration nicht zu 
erwarten ist. Der magnetische Nagellack zeigt erwartungsgemäß 
starke Absorption, da die Eisenkügelchen im Lack viele 
freie Ladungsträger besitzen, die mit der Strahlung wechselwirken
können. Die Fermienergie liegt hier im Leitungsband, was 
einen Energieübertrag jeglicher Art ermöglicht und 
damit die Wechselwirkungswahrscheinlichkeit drastisch erhöht.
Die Dicke des Lacks ist dabei entscheidend für die Menge der absorbierten Strahlung, 
was am Vergleich der magnetischen Proben gut erkennbar ist. 
Für die Silberleitlack-Probe wird eine stärkere Absorption als die 
gezeigte Erwartet, da auch hier die leitende Eigenschaft und die 
damit einhergehende Anzahl an freien Ladungsträgern dafür sorgen, 
dass die Strahlung mir hoher Wahrscheinlichkeit wechselwirkt. 
Am geringen zeitlichen Versatz ist es erkennbar, dass die 
Schicht sehr dünn aufgetragen wurde oder dass die Konzentration der 
Silberpartikel beim Auftragen sehr gering war. Die im Vergleich zu den 
anderen Proben wohl dünnere Wechselwirkungs-Schicht ist daher der Grund, 
weswegen die Intensität des Pulses deutlich weniger stark abnimmt 
als die des magnetischen Lacks. \\
Insgesamt können wir über die qualitative Analyse der THz-TDS bei den 
Lackproben einige Aussagen über die im Lack befindlichen Hauptkomponenten 
ableiten. Die Messergebnisse entsprechen den Erwartungen und sind 
zufriedenstellend.