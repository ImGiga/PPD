\subsection{\label{subsec:FZV2}Frage 2}
\textbf{\textit{Was wird als "Terahertz gap" bezeichnet?}}\\
$\rightarrow$Der "Terahertz-Gap" bezieht sich auf den Frequenzbereich im elektromagnetischen Spektrum, 
der im Allgemeinen zwischen dem Mikrowellenbereich und dem Infrarotbereich liegt. 
Genauer gesagt erstreckt sich dieser Bereich typischerweise von 
etwa $0,1\,\si{THz}$ bis etwa $10\,\si{THz}$.
Hochgeschwindigkeitstransistoren ermöglichen die effiziente Erzeugung von Strahlung im 
Niederfrequenzbereich durch Schwingkreise, während bei hohen Frequenzen Halbleiterlaser 
kohärentes Licht für Telekommunikationsverbindungen über optische Fasern erzeugen.
Die Herausforderung, die diesen Bereich als "Gap" charakterisiert, besteht darin, 
dass traditionelle Elektronik (wie Halbleiterbauelemente) nicht effizient für die Erzeugung, 
Detektion oder Manipulation von Terahertz-Strahlen genutzt werden kann \cite{Gap}. \\ 

\textbf{\textit{Wie kann er geschlossen werden?}}\\
$\rightarrow$Der Terahertz-Gap kann auf verschiedene Weisen geschlossen werden, 
wobei Forschung und Entwicklung auf diesem Gebiet aktiv voranschreiten. 
Einige Ansätze zur Überwindung des Terahertz-Gaps umfassen neue Materialien und Technologien,
Terahertz-Quellen und Detektoren, sowie Fortschritte in der Halbleiterphysik. 
Ein Ansatz von Köhler \textit{et al.} \cite{Kohler} ist der Einsatz eines Halbleiterlasers,
dessen aktive Region aus identischen Quantenstrukturen besteht, die seriell angeordnet sind. 
Im Quantenkaskaden-Schema werden Elektronen gezielt in einen angeregten Zustand injiziert, 
springen zu einem Zustand niedrigerer Energie, emittieren Licht und werden in der 
nächsten Struktur recycelt. Für Laserwirkung müssen die meisten Elektronen im angeregten Zustand bleiben, 
was hier durch eine Aneinanderreihung von über hundert gleichen Quantenstrukturen ermöglicht wird, 
die durch gezieltes Halbleiterkristallwachstum gezüchtet wurden. \\
Die Kontrolle der Geräteeigenschaften und die Flexibilität des Halbleiterkristallwachstums 
deuten darauf hin, dass Verbesserungen, die zu einer kommerziellen Terahertz-Technologie führen, 
durchaus möglich sind \cite{Gap}. \\

\textbf{\textit{Was sind die Vorteile von THz-Strahlung im Vergleich zu optischen Frequenzen?}}\\
$\rightarrow$Terahertz (THz)-Strahlung hat mehrere Vorteile im Vergleich zu optischen Frequenzen, 
die auf ihren speziellen Eigenschaften und Anwendungen basieren. 
Im Vergleich zu optischen Frequenzen kann THz-Strahlung tiefer in Materie eindringen, 
was Anwendungen in der medizinischen Bildgebung, Sicherheitsscans und Materialprüfung ermöglicht.
THz-Strahlung kann spezifische molekulare Schwingungen anregen, die charakteristisch 
für bestimmte Substanzen sind. 
Dies ermöglicht Anwendungen in der chemischen Identifikation und Spektroskopie.
Im Gegensatz zu Röntgen- oder UV-Strahlung ist THz-Strahlung nichtionisierend. 
Das bedeutet, dass sie keine ionisierende Wirkung auf biologisches Gewebe hat 
und daher in medizinischen Anwendungen, wie beispielsweise der bildgebenden Diagnostik, 
als sicherer betrachtet wird.
Im Vergleich zu optischen Frequenzen ist THz-Strahlung weniger empfindlich gegenüber 
Streuung durch kleine Partikel oder Unreinheiten in der Luft. 
Dies erleichtert den Einsatz in Umgebungen mit anspruchsvollen Bedingungen.
THz-Strahlung hat das Potenzial für drahtlose Hochgeschwindigkeitskommunikation 
aufgrund ihrer hohen Bandbreite \cite{Kohler,Gap,UltraFast}.