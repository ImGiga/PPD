\section{\label{sec:einleitung}Einleitung}
In diesem Versuch werden die Schlüsselaspekte der Ultrakurzzeit-Physik, 
insbesondere im Kontext der Femtosekunden-Autokorrelation und Terahertz-Spektroskopie, 
eingehend betrachtet. 
Diese Forschungsbereiche sind von besonderem Interesse, da sie es ermöglichen, 
Prozesse auf ultrakurzen Zeitskalen zu untersuchen, darunter atomare Schwingungen und 
Ladungsträgerdynamiken in Halbleitern. \\
Aufgrund der Begrenzungen herkömmlicher Oszilloskope mit einer zeitlichen Auflösung 
von etwa $10$ Pikosekunden greifen wir auf die Erzeugung von Femtosekunden-Pulsen zurück, 
die durch Moden-gekoppelte Laser realisiert werden. 
Im Verlauf dieser Auswertung werden die experimentellen Schritte und Ergebnisse zweier Hauptaspekte behandelt. \\
Im ersten Abschnitt liegt der Fokus auf dem Aufbau eines optischen Autokorrelators. 
Dieser ermöglicht die Bestimmung der Pulsdauer ultrakurzer Lichtpulse mittels Autokorrelation, 
einer Methode, die es gestattet, die Pulsdauer zu charakterisieren, wenn herkömmliche elektronische 
Messgeräte nicht ausreichend schnell agieren können. \\
Der zweite Abschnitt widmet sich der Nutzung von Femtosekunden-Pulsen zur Erzeugung von Terahertz-Pulsen. 
Der Terahertz-Spektralbereich, welcher Frequenzen zwischen $300$ GHz und $10$ THz abdeckt, 
eröffnet die Möglichkeit, fundamentale Anregungen in Festkörpern zu erforschen. 
Die Terahertz-Zeitdomänen-Spektroskopie (THz-TDS) erlaubt die zeitliche Abtastung des elektrischen 
THz-Feldes, um Informationen über die frequenzabhängige Absorption und den Brechungsindex zu gewinnen. \\
Im Verlauf dieses Praktikumsberichts werden die theoretischen Grundlagen, der experimentelle Aufbau, 
die durchgeführten Messungen und die daraus resultierenden Erkenntnisse detailliert behandelt. 
Ziel ist es, einen umfassenden Überblick über die angewandten Methoden und die gewonnenen Ergebnisse zu geben, 
um die Komplexität der Ultrakurzzeit-Physik in den vorgestellten Experimenten zu verdeutlichen.