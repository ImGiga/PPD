\subsection{\label{sec:A23}p-dotiertes Silizium}
Im Vergleich zum hochohmigen Silizium wird ein THz-Puls aufgenommen, der durch einen Wafer aus 
p-dotierten Silizium (p-Si) propagiert. Die Aufnahmeparameter werden wie folgt gewählt
\begin{equation}
    \Delta s = 0,001\,\si{mm}, \hspace{1.5cm} N_{\text{s}} = 700, \hspace{1.5cm} N_{\text{sample}} = 100, \hspace{1.5cm} s_{0} = 153,3\,\si{mm},
\end{equation} 
was die schnelle Aufnahme des Hauptpulses ermöglicht. Der Aufgenommene Puls ist im Vergleich zum Referenzpuls in 
Abb.~\ref{fig:pdot} dargestellt. Es ist zu beachten, dass sich die Verzögerung $\Delta\tau$ auf zwei verschiedene 
Startpunkte bezieht, weswegen die gezeigten Pulse nur in ihrer Form und nicht in ihrer Position vergleichbar sind. \\ 
Im Unterschied zum HR-Si wird die Transmission beim Durchgang durch das p-Si mit dem Drude-Modell berechnet.
Grundlage hierfür ist die erhöhte Anzahl an freien Ladungsträgern, die mit dem THz-Feld 
wechselwirken und mit einer gedämpften Schwingung, der Plasma-Frequenz $\omega_{\text{p}}$ und 
Dämpfung $\Gamma$ reagieren. Für das HR-Si konnte diese Schwingung aufgrund der hohen 
Resistivität vernachlässigt werden. Es gelten folgende Zusammenhänge
\begin{align}
    T(\omega)_{\text{Drude}} &= \frac{4n(\omega)}{(n(\omega)+1)^{2}}e^{-i(n(\omega)-1)\omega L/c} \\
    n(\omega) &= \sqrt{\epsilon_{\infty} - \frac{\omega_{\text{p}}^{2}}{\omega^{2}-i\Gamma\omega}}.
\end{align}
Hierbei entspricht $c$ der Lichtgeschwindigkeit im Vakuum und $L$ gibt die Dicke des 
Wafers an, für die wir $L = (380\pm5)\,\si{\mu m}$ messen.
Der theoretische Verlauf kann an die gemessene Transmission, 
welche analog zum vorherigen Abschnitt berechnet wird, 
über die Anpassung von $\omega_{\text{p}}$ und $\Gamma$ gefittet werden. 
Hieraus lassen sich Materialeigenschaften wie die Ladungsträgermobilität $\mu$, 
die Ladungsträgerdichte $N$, die Leitfähigkeit $\sigma$ und der Widerstand 
$\rho = \sigma^{-1}$ berechnen.